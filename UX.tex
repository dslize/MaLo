\documentclass[11pt, a4paper]{article}

\usepackage{graphicx} 
\usepackage[utf8]{inputenc}
\usepackage{fancyhdr}
\usepackage{changepage}
\usepackage[onehalfspacing]{setspace}
\usepackage{ragged2e}
\usepackage{ amssymb, amsmath, amsthm, dsfont }
\usepackage[width = 18cm, top = 2.5cm, bottom = 3cm]{geometry}
\usepackage{extarrows}
% ---------

\newcommand{\myTitleString} {}
\newcommand{\myAuthorString} {}
\newcommand{\mySubTitleString} {}
\newcommand{\myDateString} {}

\newcommand{\myTitle}[1] {\renewcommand {\myTitleString}{#1}}
\newcommand{\mySubTitle}[1] {\renewcommand {\mySubTitleString}{#1}}
\newcommand{\myAuthor}[1] {\renewcommand{\myAuthorString}		{#1}}
\newcommand{\myDate}[1] {\renewcommand{\myDateString}{#1}}

\newcommand{\makeMyTitle}
{
\pagestyle{fancy}
\fancyhead[L]
{
\begin{tabular}{l}
\myTitleString
\\ \mySubTitleString 
\\ \myDateString
\end{tabular}
} 			
\fancyhead[C]{}
\fancyhead[R]{\myAuthorString}
\fancyfoot[C]{\thepage}
}

\setlength{\headheight}{45pt}

\newcommand{\p}{11}
\newcommand{\pp}{6}
\newcommand{\ppp}{6}
\newcommand{\pppp}{} 


\makeatletter
\renewcommand*\env@matrix[1][*\c@MaxMatrixCols c]{%
  \hskip -\arraycolsep
  \let\@ifnextchar\new@ifnextchar
  \array{#1}}
\makeatother
% ---------
%\setlength{\parindent}{0pt}
\begin{document}

\myTitle{Mathematische Logik}
\mySubTitle{Übung X}
\myDate{24. April 2017}
\myAuthor
{
\begin{tabular}{l l}
346532, &Daniel Boschmann\\
348776, &Anton Beliankou	\\
356092, &Daniel Schleiz
\end{tabular}
}
\makeMyTitle

\begin{tabular}{|c|c|c|c|}\hline
   1 & 2 & 3 & $\sum$\\\hline
  	 \qquad/\p & \qquad/\pp & \qquad/\ppp & \qquad/23\\\hline
 \end{tabular}
\hspace*{20pt} Korrigiert am:\underline{\hspace{3cm}}
\vspace*{30pt}


\section*{Aufgabe 1 (Punkte:\qquad/\p)}
\textbf{(a)}
\begin{adjustwidth}{20pt}{20pt}


\end{adjustwidth}
\textbf{(b)}
\begin{adjustwidth}{20pt}{20pt}

\end{adjustwidth}
\textbf{(c)}
\begin{adjustwidth}{20pt}{20pt}



\end{adjustwidth}




\section*{Aufgabe 2 (Punkte:\qquad/\pp)}

\begin{adjustwidth}{20pt}{20pt}

\end{adjustwidth}





\section*{Aufgabe 3 (Punkte:\qquad/\ppp)}

\begin{adjustwidth}{20pt}{20pt}

\end{adjustwidth}


\end{document}
