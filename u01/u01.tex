\documentclass[11pt, a4paper]{article}

\usepackage{graphicx} 
\usepackage[utf8]{inputenc}
\usepackage{fancyhdr}
\usepackage{changepage}
\usepackage[onehalfspacing]{setspace}
\usepackage{ragged2e}
\usepackage{ amssymb, amsmath, amsthm, dsfont }
\usepackage[width = 18cm, top = 2.5cm, bottom = 3cm]{geometry}
\usepackage{extarrows}
\usepackage{ stmaryrd }
% ---------

\newcommand{\myTitleString} {}
\newcommand{\myAuthorString} {}
\newcommand{\mySubTitleString} {}
\newcommand{\myDateString} {}

\newcommand{\myTitle}[1] {\renewcommand {\myTitleString}{#1}}
\newcommand{\mySubTitle}[1] {\renewcommand {\mySubTitleString}{#1}}
\newcommand{\myAuthor}[1] {\renewcommand{\myAuthorString}		{#1}}
\newcommand{\myDate}[1] {\renewcommand{\myDateString}{#1}}

\newcommand{\makeMyTitle}
{
\pagestyle{fancy}
\fancyhead[L]
{
\begin{tabular}{l}
\myTitleString
\\ \mySubTitleString 
\\ \myDateString
\end{tabular}
} 			
\fancyhead[C]{}
\fancyhead[R]{\myAuthorString}
\fancyfoot[C]{\thepage}
}

\setlength{\headheight}{45pt}

\newcommand{\p}{8}
\newcommand{\pp}{9}
\newcommand{\ppp}{10}
\newcommand{\pppp}{10*} 

\newcommand{\defgl}{\mathrel{=\!\!\mathop:}}
\newcommand{\defgr}{\mathrel{\mathop:\!\!=}}

\makeatletter
\renewcommand*\env@matrix[1][*\c@MaxMatrixCols c]{%
  \hskip -\arraycolsep
  \let\@ifnextchar\new@ifnextchar
  \array{#1}}
\makeatother

\newcommand{\interpretation}[1] {\mathfrak{#1}} 
% ---------
%\setlength{\parindent}{0pt}
\begin{document}

\myTitle{Mathematische Logik}
\mySubTitle{Übung 1}
\myDate{24. April 2017}
\myAuthor
{
\begin{tabular}{l l}
356092, &Daniel Schleiz\\
348776, &Anton Beliankou	\\
346532, &Daniel Boschmann
\end{tabular}
}
\makeMyTitle

\begin{tabular}{|c|c|c|c|c|}\hline
   3 & 4 & 5 & 6 &$\sum$\\\hline
  	 \qquad/\p & \qquad/\pp & \qquad/\ppp & \qquad/\pppp & \qquad/27\\\hline
 \end{tabular}
\hspace*{20pt} Korrigiert am:\underline{\hspace{3cm}}
\vspace*{30pt}


\section*{Aufgabe 3 (Punkte:\qquad/\p)}
\textbf{(a)}
\begin{adjustwidth}{20pt}{20pt}
Definiere die Variablenmenge $\{A,B,C,D,E \}$, welche jeweils die Wahrheitswerte $0$ und $1$ annehmen könnte.\\

Zur Semantik: Variable $A$ bezieht sich auf Antonia, $B$ auf Benjamin, $C$ auf Claudius, $D$ auf Desir\'{e}e, $E$ auf Emil.\\
Der Wahrheitswert $1$ bei einer der Variablen bedeutet, dass die dadurch referenzierte Person zum Essen erscheint, $0$ bedeutet dagegen, die Person kommt nicht.
\end{adjustwidth}
\textbf{(b)}
\begin{adjustwidth}{20pt}{20pt}
(i) $\varphi_1 \defgr \neg A \rightarrow  \neg B$ \hspace*{60pt} (iv) $\varphi_4 \defgr (E\wedge D)\vee(\neg E\wedge \neg D)$\\
(ii) $\varphi_2\defgr C \vee D$ \hspace*{78pt}  (v) $\varphi_5 \defgr C \rightarrow (D\wedge B)$\\
(iii) $\varphi_3\defgr (A\wedge\neg E)\vee(\neg A \wedge E)$\\

Aus den Teilformeln ergibt sich die Gesamtformel:\\
(Konjunktion der Teilformeln, da alle Bedingungen (i)-(v) erfüllt werden müssen)\\

$\varphi=\bigwedge\limits_{i=1}^5 \varphi_i$
\end{adjustwidth}
\textbf{(c)}
\begin{adjustwidth}{20pt}{20pt}
Angenommen, es existiert eine passende Interpretation $\interpretation{I}:\{A,B,C,D,E \}\mapsto \{0,1\}$ mit $\llbracket \varphi \rrbracket^{\interpretation{I}}=1$, welche $C$ auf $1$ abbildet ("Claudius kommt"). Damit die Teilformel $\varphi_5$ wahr ist, muss gelten:\\

$1\rightarrow (D\wedge B) \equiv \neg 1\vee (D\wedge B) \equiv 0 \vee (D\wedge B) \equiv (D\wedge B)$\\

Das heißt, $\interpretation{I}$ muss $D\mapsto 1$ und $B\mapsto 1$ abbilden. Nun folgt, dass auch $E\mapsto 1$ gelten muss für $\interpretation{I}$, denn aus $\varphi_4$ ergibt sich:\\

$(E\wedge 1)\vee(\neg E\wedge \neg 1) \equiv E\vee(\neg E\wedge 0) \equiv E\vee 0 \equiv E$ \qquad (da $\llbracket \varphi \rrbracket^{\interpretation{I}}=1$)\\

Ferner gilt für $\varphi_3$ beim Einsetzen:\\

$(A\wedge\neg 1)\vee(\neg A \wedge 1) \equiv (A\wedge 0)\vee \neg A  \equiv 0\vee \neg A  \equiv \neg A$\\

Damit $\varphi_3$ wahr ist, muss $\interpretation{I}$ also $A$ auf $0$ abbilden. Setze nun in $\varphi_1$ ein:\\

$\neg 0 \rightarrow  \neg 1 \equiv 1 \rightarrow  0 \equiv \neg 1 \vee 0 \equiv 0 \vee 0 \equiv 0$ (Widerspruch! $\lightning$)\\

Da die Teilformel $\varphi_1$ nicht erfüllt ist, war die Annahme falsch, dass ein Modell $\interpretation{I}$ mit $C\mapsto 1$ existiert.\\

Falls also für $\varphi$ ein Modell $\interpretation{I}$ existiert, so muss $\interpretation{I}$ $C\mapsto 0$ abbilden.\\
Setze in $\varphi_2$ ein:\\

$0\vee D \equiv D$\\

Es folgt, dass $D\mapsto 1$ gelten muss.\\
Setze ein in $\varphi_4$:\\

$(E\wedge 1)\vee(\neg E\wedge \neg 1) \equiv E \vee (\neg E\wedge 0) \equiv E\vee 0 \equiv E$\\

Somit muss auch $E\mapsto 1$ gelten, da jede Teilformel wahr sein muss.\\
Setze in $\varphi_3$ ein:\\

$(A\wedge\neg 1)\vee(\neg A \wedge 1)\equiv (A\wedge 0)\vee(\neg A \wedge 1) \equiv 0 \vee \neg A \equiv \neg A$\\

Damit folgt $A\mapsto 0$. Betrachte nun $\varphi_1$:\\

$\neg 0 \rightarrow  \neg B \equiv 1 \rightarrow  \neg B \equiv \neg 1 \vee \neg B \equiv 0 \vee \neg B  \equiv \neg B$\\

Daraus folgt $B\mapsto 0$.\\

Es wurden alle möglichen Belegungen von $C$ betrachtet, die restlichen Belegungen folgen aus $C\mapsto 0$ in $\interpretation{I}$. Somit existiert genau ein Modell $\interpretation{I}$, mit $A\mapsto 0, B \mapsto 0, C \mapsto 0, D \mapsto 1, E\mapsto 1$.\\

Es kommen also nur Desir\'{e}e und Emil zum Essen.


\end{adjustwidth}




\section*{Aufgabe 4 (Punkte:\qquad/\pp)}

\textbf{(a)}
\begin{adjustwidth}{20pt}{20pt}
(i)\\
 $(X\rightarrow 1)\rightarrow(0\rightarrow Y)\\ \equiv (X\rightarrow 1)\rightarrow(\neg 0\vee Y)\\ \equiv (X\rightarrow 1)\rightarrow(1 \vee Y)\\ \equiv (X\rightarrow 1)\rightarrow 1\\ \equiv \neg (X \rightarrow 1) \vee 1\equiv 1$\\

Die Formel ist eine Tautologie (stets wahr) und damit insbesondere erfüllbar.\\
(ii)\\
 $(1 \rightarrow X \vee Y) \wedge (0 \rightarrow (\neg X \wedge \neg Y))\\ \equiv (1 \rightarrow X \vee Y) \wedge (\neg 0 \vee (\neg X \wedge \neg Y))\\ \equiv (1 \rightarrow X \vee Y) \wedge (1 \vee (\neg X \wedge \neg Y))\\ \equiv (1 \rightarrow X \vee Y) \wedge 1 \equiv (1 \rightarrow X \vee Y)\\ \equiv \neg 1 \vee (X \vee Y) \equiv 0 \vee (X \vee Y) \equiv X \vee Y$\\

Die Formel ist erfüllbar, da z. B. mit Interpretation $\interpretation{I}$ mit $X\mapsto 1$, $Y\mapsto 0$ ein Modell ist, aber keine Tautologie, denn die Interpretation $\interpretation{J}$ mit $X\mapsto 0$, $Y\mapsto 0$ führt zum Wahrheitswert $0$ der Formel.\\
(iii)\\
 $\neg (Y \rightarrow X)\wedge(\neg X \rightarrow (X \wedge Y))\\ \equiv \neg (\neg Y \vee X)\wedge(\neg X \rightarrow (X \wedge Y))\\ \equiv ( Y \wedge \neg X)\wedge(\neg X \rightarrow (X \wedge Y))\\ \equiv ( Y \wedge \neg X)\wedge( X \vee (X \wedge Y))\\ \equiv ( Y \wedge \neg X)\wedge X \equiv Y \wedge \neg X \wedge X$\\

Die Formel ist unerfüllbar, da $X$ in einer Interpretation nur einen Wahrheitswert $1$ oder $0$ besitzen darf, somit lässt sich $\neg X \wedge X$ nicht erfüllen und die Formel besitzt für alle Belegungen von $X$ und $Y$ einen Wahrheitswert $0$.

\end{adjustwidth}
\textbf{(b)}
\begin{adjustwidth}{20pt}{20pt}
(i)\\
 $(Y \rightarrow X)\wedge((X \vee Z)\rightarrow Y) \equiv (\neg Y \vee X)\wedge(\neg(X \vee Z)\vee Y)\\ \overset{\text{Distr.}}\equiv ((\neg Y \vee X)\wedge Y) \vee ((\neg Y \vee X)\wedge \neg(X \vee Z))\\
\overset{\text{Distr.}}\equiv ((\neg Y \wedge Y) \vee (X \wedge Y)) \vee ((\neg Y \vee X)\wedge \neg(X \vee Z))\\ \equiv (0 \vee (X \wedge Y)) \vee ((\neg Y \vee X)\wedge \neg(X \vee Z))\\ \equiv (X \wedge Y) \vee ((\neg Y \vee X)\wedge \neg(X \vee Z))\\ \equiv (X \wedge Y) \vee ((\neg Y \vee X)\wedge (\neg X \wedge \neg Z))\\ \overset{\text{Distr.}}\equiv (X \wedge Y) \vee ((\neg Y \wedge  \neg X \wedge \neg Z)\vee (X \wedge \neg X \wedge \neg Z))\\ \equiv (X \wedge Y) \vee ((\neg Y \wedge  \neg X \wedge \neg Z)\vee 0)\\ \equiv (X \wedge Y) \vee (\neg Y \wedge  \neg X \wedge \neg Z)\\ \overset{\text{De Morg.}}\equiv (X \wedge Y) \vee \neg( Y \vee   X \vee  Z)\\ \equiv ( Y \vee   X \vee  Z) \rightarrow (X \wedge Y)$\\
(ii)\\
 $(X \wedge Z) \vee ((X \wedge Z)\wedge ((Y \vee \neg Z)\rightarrow U))\\ \overset{*}\equiv (X \wedge Z) \equiv  (X \wedge Z) \wedge (X \wedge Z)\\ \overset{\text{Absorp. rückw.}}\equiv (((\neg Y \wedge Z)\vee U)\vee(X \wedge Z))\wedge(X \wedge Z)\\ \overset{\text{Assoz.}}\equiv ((\neg Y \wedge Z)\vee U \vee(X \wedge Z))\wedge X \wedge Z$\\\\
 
 *Absorption\\
 $\varphi \defgr (X \wedge Z)$\\
 $\psi \defgr ((Y \vee \neg Z)\rightarrow U)$\\
 $\varphi \vee (\varphi \wedge \psi) \equiv \varphi$
\end{adjustwidth}



\section*{Aufgabe 5 (Punkte:\qquad/\ppp)}

\textbf{(a)}
\begin{adjustwidth}{20pt}{20pt}
Für unendliche Formelmengen $\Phi$ ist $\tau(\Phi)=\bigcup\limits_{\varphi \in \Phi} \tau(\varphi)$ nicht zwingend endlich, d. h. es können unendlich viele Aussagenvariablen in den Formeln von $\Phi$ vorhanden sein.\\

Da somit in $\bigwedge\limits_{\varphi \in \Phi} \varphi$ und $\bigvee\limits_{\varphi \in \Phi} \varphi$ ebenfalls unendlich viele Aussagenvariablen vorkommen, hadelt es sich dabei nicht um aussagenlogische Formeln, da diese induktiv definiert sind und man $\bigwedge\limits_{\varphi \in \Phi} \varphi$ und $\bigvee\limits_{\varphi \in \Phi} \varphi$ nicht mit Hilfe der induktiven Regeln (in einer endlicher Anzahl von Schritten) herleiten kann, da potenziell unendlich viele Aussagenvariablen vorkommen, und man die entsprechende Basisregel (jede Aussagenvariable ist eine aussagenlogische Formel) unendlich oft anwenden müsste.

\end{adjustwidth}
\textbf{(b)}
\begin{adjustwidth}{20pt}{20pt}
(i)\\
In dem Fall von 4 Studierenden ($\{1,...,4\}$) existieren die Aussagenvariablen $X_{1,2},X_{1,3},X_{1,4},X_{2,3},X_{2,4},X_{3,4}$\\
Sei $\varphi=X_{1,2} \vee X_{1,3} \vee X_{1,4} \vee X_{2,3} \vee X_{2,4} \vee X_{3,4}$\\
Der Sachverhalt in der Aussagenlogik:\\
Es existiert eine Interpretation $\interpretation{I}_P$, sodass gilt: $\llbracket \varphi \rrbracket^{\interpretation{I}_P}=1$.\\
(ii)\\

 $\bigwedge\limits_{i=1}^n \left(\bigvee\limits_{j=i+1}^n \neg X_{i,j} \vee \bigvee\limits_{k=1}^{i-1} \neg X_{k,i} \right)$\\
 
 Konstruktion: Für jede Person (äußere Konjunktion von $i=1$ bis $n$) muss in einem Modell eine Variable mit dem Index $i$ (an erster oder zweiter Stelle) auf $0$ abgebildet werden, d. h. es existiert eine Person, die mit dieser nicht zusammenarbeiten möchte.
\end{adjustwidth}

\section*{Aufgabe 6 (Punkte:\qquad/\pppp)}

\textbf{(a)}
\begin{adjustwidth}{20pt}{20pt}
Sei $\psi(X,Y) \rightarrow \varphi(Y,Z)$ eine Tautologie. Zeige, dass $\psi(\psi(1,Y),Y)$ eine Interpolante für diese Formel ist.\\

Beobachte, dass $\tau(\psi(\psi(1,Y),Y))=\{Y\} \subseteq \{X,Y\}\cap \{Y,Z\} = \tau(\psi(X,Y))\cap \tau(\phi(Y,Z))$\\

Zeige zunächst, dass $\psi(X,Y) \rightarrow \psi(\psi(1,Y),Y)$ eine Tautologie ist. Betrachte zuerst die Interpretationen, für die gilt $\llbracket \psi(1,Y) \rrbracket=1$ (Belegung für $Y$ unerheblich).\\

Dann gilt:\\
 $\psi(X,Y) \rightarrow \psi(\psi(1,Y),Y)\\ \equiv \psi(X,Y) \rightarrow \psi(1,Y)\\ \equiv \psi(X,Y) \rightarrow 1 \equiv \neg \psi(X,Y) \vee 1 \equiv 1$ \qquad ,also stets wahr.\\
 
 Betrachte nun die Interpretationen mit $\llbracket \psi(1,Y) \rrbracket=0$ (Belegung für $Y$ unerheblich). Angenommen, für eine solche Interpretation gilt $\llbracket \psi(0,Y) \rrbracket=1$. Dann gilt:\\
 $\varphi(X,Y) \rightarrow \varphi(\varphi(1,Y),Y)\\ \equiv \psi(X,Y) \rightarrow \psi(0,Y)\\ \equiv \psi(X,Y) \rightarrow 1 \equiv \neg \psi(X,Y) \vee 1 \equiv 1$ \qquad ,also stets wahr.\\
 
 Ansonsten, existiert noch der Fall, dass $\llbracket \psi(0,Y) \rrbracket=0$. Dann gilt:\\
 
$\psi(X,Y) \rightarrow \varphi(\varphi(1,Y),Y)\\ \equiv \psi(X,Y) \rightarrow \psi(0,Y)\\ \equiv \psi(X,Y) \rightarrow 0 \equiv \neg \psi(X,Y) \vee 0 \equiv \neg \psi(X,Y) \equiv \neg 0 \equiv 1$ \qquad ,da angenommen, dass sowohl $\llbracket \psi(1,Y) \rrbracket=0$, als auch $\llbracket \psi(0,Y) \rrbracket=0$, also immer.\\

Da alle Fälle betrachtet wurden und $\psi(X,Y) \rightarrow \psi(\psi(1,Y),Y)$ für alle Interpretationen den Wahrheitswert $1$ besitzt, handelt es sich um eine Tautologie.\\

Nun gilt noch zu zeigen ,dass $\psi(\psi(1,Y),Y) \rightarrow \varphi(X,Y)$ eine Tautologie ist. Per Annahme ist $\psi(X,Y)\rightarrow \varphi(Y,Z)$ eine Tautologie, d. h. die Belegung von $X$ ist für den Wahrheitsgehalt der Formel unerheblich, dieser ist stets $1$ für eine beliebige Interpretation. Somit lässt sich $X$ durch eine Formel, wie z.B. $\psi(1,Y)$ ersetzen und es handelt sich dennoch um eine Tautologie.\\

Insgesamt folgt nun, dass $\psi(\psi(1,Y),Y)$ eine Interpolante für $\psi(X,Y) \rightarrow \varphi(Y,Z)$ ist.
 
\end{adjustwidth}

\end{document}
