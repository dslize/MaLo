\documentclass[11pt, a4paper]{article}

\usepackage{graphicx} 
\usepackage[utf8]{inputenc}
\usepackage{fancyhdr}
\usepackage{changepage}
\usepackage[onehalfspacing]{setspace}
\usepackage{ragged2e}
\usepackage{ amssymb, amsmath, amsthm, dsfont }
\usepackage[width = 18cm, top = 2.5cm, bottom = 3cm]{geometry}
\usepackage{extarrows}
\usepackage{stmaryrd}
% ---------

\newcommand{\myTitleString} {}
\newcommand{\myAuthorString} {}
\newcommand{\mySubTitleString} {}
\newcommand{\myDateString} {}

\newcommand{\myTitle}[1] {\renewcommand {\myTitleString}{#1}}
\newcommand{\mySubTitle}[1] {\renewcommand {\mySubTitleString}{#1}}
\newcommand{\myAuthor}[1] {\renewcommand{\myAuthorString}		{#1}}
\newcommand{\myDate}[1] {\renewcommand{\myDateString}{#1}}

\newcommand{\defgl}{\mathrel{=\!\!\mathop:}}
\newcommand{\defgr}{\mathrel{\mathop:\!\!=}}
\newcommand{\makeMyTitle}
{
\pagestyle{fancy}
\fancyhead[L]
{
\begin{tabular}{l}
\myTitleString
\\ \mySubTitleString 
\\ \myDateString
\end{tabular}
} 			
\fancyhead[C]{}
\fancyhead[R]{\myAuthorString}
\fancyfoot[C]{\thepage}
}

\setlength{\headheight}{45pt}

\newcommand{\p}{10}
\newcommand{\pp}{10}
\newcommand{\ppp}{20}
\newcommand{\pppp}{} 


\makeatletter
\renewcommand*\env@matrix[1][*\c@MaxMatrixCols c]{%
  \hskip -\arraycolsep
  \let\@ifnextchar\new@ifnextchar
  \array{#1}}
\makeatother
% ---------
%\setlength{\parindent}{0pt}
\begin{document}

\myTitle{Mathematische Logik}
\mySubTitle{Übung 2}
\myDate{1. Mai 2017}
\myAuthor
{
\begin{tabular}{l l}
346532, &Daniel Boschmann\\
348776, &Anton Beliankou	\\
356092, &Daniel Schleiz
\end{tabular}
}
\makeMyTitle

\begin{tabular}{|c|c|c|}\hline
    2 & 3 & $\sum$\\\hline
  	 \qquad/\pp & \qquad/\ppp & \qquad/30\\\hline
 \end{tabular}
\hspace*{20pt} {\huge Gruppe \textbf{G}}
\vspace*{30pt}


\section*{Aufgabe 2 (Punkte:\qquad/\p)}
\textbf{(a)}
\begin{adjustwidth}{20pt}{20pt}
(i)\\
TODO: Zu zeigen: $\{\neg , \leftrightarrow \}$ ist nicht funktional vollständig.\\
(ii)\\
Zu zeigen: $\{\downarrow\}$ ist funktional vollständig.\\

Aus der Vorlesung (Def. 1.10) ist bekannt,  $\{ \vee , \neg \}$ ist funktional vollständig.\\
$\neg A \equiv \neg A \wedge \neg A \equiv A \downarrow A$\\
$A \vee B \equiv \neg \neg (A \vee B) \equiv \neg (\neg A \wedge \neg B) \equiv \neg (A \downarrow B) \equiv (A \downarrow B)\downarrow(A \downarrow B)$\\
Somit folgt die zu zeigende Aussage.
\end{adjustwidth}
\textbf{(b)}
\begin{adjustwidth}{20pt}{20pt}
Zu zeigen: $\{f, 0, 1\}$ ist funktional vollständig.\\

Aus der Vorlesung (Def. 1.10) ist bekannt,  $\{ \wedge , \neg \}$ ist funktional vollständig.\\
$\neg A \equiv f(A,1,0)$\\
$A \wedge B \equiv f(A,0,B)$\\
Somit folgt die zu zeigende Aussage.
\end{adjustwidth}




\section*{Aufgabe 3 (Punkte:\qquad/\ppp)}

\textbf{(a)}
\begin{adjustwidth}{20pt}{20pt}
Sei $\varphi \defgr (A \wedge C \rightarrow B)\wedge(F \wedge D \rightarrow H)\wedge(D \wedge C \wedge E \rightarrow F)\wedge(B \wedge C \rightarrow E)\wedge(1 \rightarrow A)\wedge(H \rightarrow 0)\wedge(1 \rightarrow C)\wedge(A \rightarrow D)$\\

Schritt 0: Markiere Variablen $A$, für welche Klauseln $1 \rightarrow A$ existieren:\\
\begin{center}
	$M = \{A,C\}$\\
\end{center}
\vspace*{15 pt}

Schritt 1: Markiere Variablen $B$ und $D$ wegen jeweils $A \wedge C \rightarrow B$ und  $A \rightarrow D$ mit $A,C \in M$:\\
\begin{center}
	$M = \{A, B, C, D\}$\\
\end{center}
\vspace*{15 pt}

Schritt 2: Markiere Variable $E$ wegen  $B \wedge C \rightarrow E$ mit $B,C \in M$:\\
\begin{center}
	$M = \{A, B, C, D, E\}$\\
\end{center}
\vspace*{15 pt}
Schritt 3: Markiere Variable $F$ wegen  $D \wedge C \wedge E \rightarrow F$ mit $
D,C,E \in M$:\\
\begin{center}
	$M = \{A, B, C, D, E, F\}$\\
\end{center}
\vspace*{15 pt}
Schritt 4: Markiere Variable $H$ wegen  $F \wedge D \rightarrow H$ mit $
F,D \in M$:\\
\begin{center}
	$M = \{A, B, C, D, E, F, H\}$\\
\end{center}
\vspace*{15 pt}

Wegen Klausel $H \rightarrow 0$ mit $H \in M$ ist die Ausgabe "unerfüllbar".
\end{adjustwidth}
\textbf{(b)}
\begin{adjustwidth}{20pt}{20pt}
(i)\\
\textbf{Schnitt}\\
Die Horn-Formel haben folgenden Gestalt: $\varphi = \bigwedge \limits_{i=1}^n \psi_i$, wobei $\psi_i= (\bigvee \limits_{j=1}^k\neg X_{ij}) \vee X_{i0}$ oder $\psi_i= (\bigvee \limits_{j=1}^k\neg X_{ij})$\\

Offenbar gilt für jede Interpretation $\mathfrak{I}$:\\

$\mathfrak{I} \models \varphi \Leftrightarrow \mathfrak{I} \models \psi_i$ für $\forall i \in [1,n]$\\

Für zwei Interpretationen $\mathfrak{I_1} \models \psi_i$ und $\mathfrak{I_2} \models \psi_i$ muss  gelten $(\mathfrak{I_1} \cap \mathfrak{I_2}) \models \psi_i$, falls die Abgeschlossenheit unter der Schnittbildung gilt.\\

Dazu kann es zwei Fälle für jedes $i$ geben:\\
\textbf{Fall 1)}Für alle $j$ gilt: $\mathfrak{I_1}(X_{ij})=1 \wedge \mathfrak{I_2}(X_{ij})=1$  \hspace*{45 pt} $\Rightarrow (\mathfrak{I_1} \cap \mathfrak{I_2})(X_{ij})=1$\\
\textbf{Fall 2)}Es existiert ein $j$ mit $\mathfrak{I_1}(X_{ij})=0 \vee \mathfrak{I_2}(X_{ij})=0$ \hspace*{15 pt} $\Rightarrow (\mathfrak{I_1} \cap \mathfrak{I_2})(X_{ij})=0$\\

Für beide Fälle gilt nach der Definition $(\mathfrak{I_1} \cap \mathfrak{I_2}) = \min(\mathfrak{I_1}, \mathfrak{I_2})$: falls $\mathfrak{I_1} \models \psi_i$ und $\mathfrak{I_2} \models \psi_i$, dann $(\mathfrak{I_1} \cap \mathfrak{I_2}) \models \psi_i$.\\

Damit sind Modelle von Horn-Formeln unter Schnittbildung abgeschlossen.\\
(ii)\\
\textbf{Vereinigung}\\
Sei $\varphi(A,B) = \neg A \vee \neg B  \equiv A \wedge B \rightarrow 0$ eine Horn-Formel und $\mathfrak{I_1}$ und $\mathfrak{I_2}$ Interpretationen mit\\
$\mathfrak{I_1}(A)=0$, $\mathfrak{I_1}(B)=1$\\
$\mathfrak{I_2}(A)=1$, $\mathfrak{I_2}(B)=0$\\
$\Rightarrow \mathfrak{I_1} \models \varphi, \mathfrak{I_2} \models \varphi$\\

Ferner gilt:\\ 
$(\mathfrak{I_1} \cup \mathfrak{I_2})(A)= \max(\mathfrak{I_1}(A),\mathfrak{I_2}(A)) = \max(0,1) = 1$\\
$(\mathfrak{I_1} \cup \mathfrak{I_2})(B)= \max(\mathfrak{I_1}(B),\mathfrak{I_2}(B)) = \max(1,0) = 1$\\
$\Rightarrow (\mathfrak{I_1} \cup \mathfrak{I_2}) \not\models \varphi$ \qquad Widerspruch!\\

Das $(\mathfrak{I_1} \cup \mathfrak{I_2})$ ist also kein Modell von $\varphi$. Damit sind Modelle von Horn-Formeln unter Vereinigung nicht abgeschlossen.\\ 
(iii)\\
\textbf{Komplement}\\
Sei $\varphi(A)=A$ eine Horn-Formel sowie eine Interpretation $\mathfrak{I}(A)=1$.\\
 Es gilt offensichtlich $\mathfrak{I} \models \varphi$. Betrachte nun das Komplement von $\mathfrak{I}$:\\
 $\neg \mathfrak{I} (A) \equiv 1 - \mathfrak{I}(A) = 1- 1=0 \Rightarrow \neg \mathfrak{I} \not\models \varphi$ \qquad Widerspruch!\\
 
 Das $\neg \mathfrak{I}$ ist also kein Modell von $\varphi$. Damit sind Modelle von Horn-Formeln unter Komplement nicht abgeschlossen. 
\end{adjustwidth}
\textbf{(c)}
\begin{adjustwidth}{20pt}{20pt}
Die Umkehrung gilt nicht.\\

Betrachte $\varphi \defgr \neg A \vee B \vee C$.\\
 Das eindeutige kleinste Modell von $\varphi$ ist $\mathfrak{I_1}$ mit $\mathfrak{I_1}(X)=0,\mathfrak{I_1}(Y)=0,\mathfrak{I_1}(Z)=0$\\
 
 Sei $\mathfrak{I_2}$ mit $\mathfrak{I_2}(A)=1,\mathfrak{I_2}(B)=0,\mathfrak{I_2}(C)=1$\\
 sowie $\mathfrak{I_3}$ mit $\mathfrak{I_3}(A)=1,\mathfrak{I_3}(B)=1,\mathfrak{I_3}(C)=0$\\ zwei weitere Interpretationen.\\
 
 $(\mathfrak{I_2} \cap \mathfrak{I_3})(A)=1,(\mathfrak{I_2} \cap \mathfrak{I_3})(B)=0,\mathfrak{I_3})(C)=0$\\
 
  Es gilt $\mathfrak{I_2}  \models \varphi,\mathfrak{I_3} \models \varphi$, aber dennoch $(\mathfrak{I_2} \cap \mathfrak{I_3}) \not \models \varphi$. Somit ist $\varphi$ nicht äquivalent zu einer Horn-Formel nach Teilaufgabe b). 

\end{adjustwidth}
\textbf{(d)}
\begin{adjustwidth}{20pt}{20pt}
(i)\\
$(Z \rightarrow (X \vee \neg Y))\wedge (X \rightarrow (\neg Y \vee \neg Z))\wedge\neg(X \rightarrow (\neg Y \wedge U))$\\
$\equiv (\neg Z \vee (X \vee \neg Y))\wedge (\neg X \vee (\neg Y \vee \neg Z))\wedge\neg(\neg X \vee (\neg Y \wedge U))$\\
$\equiv (\neg Z \vee X \vee \neg Y)\wedge (\neg X \vee \neg Y \vee \neg Z)\wedge  X \wedge  (Y \vee \neg U)$\\

Die Formel ist äquivalent zu einer Horn-Formel.\\
(ii)\\
$(((\neg U \wedge (Y \vee \neg X)) \rightarrow Z) \vee (X \wedge (\neg U \rightarrow U))$\\
$\equiv (\neg((\neg U \wedge (Y \vee \neg X)) \vee Z) \vee (X \wedge U)$\\
$\equiv (U \vee (\neg Y \wedge X) \vee Z) \vee (X \wedge U)$\\

Wähle nun $\mathfrak{I_1}$ mit $\mathfrak{I_1}(U)=1$, $\mathfrak{I_1}(X)=\mathfrak{I_1}(Y)=\mathfrak{I_1}(Z)=0$\\
sowie $\mathfrak{I_2}$ mit $\mathfrak{I_2}(Z)=1$, $\mathfrak{I_2}(X)=\mathfrak{I_2}(Y)=\mathfrak{I_1}(U)=0$\\
Beide Interpretationen sind Modelle für die gegebene Formel.\\
Nach Teilaufgabe (b) sind die Modelle der Horn-Formeln unter Schnittbildung abgeschlossen.\\
$(\mathfrak{I_1} \cap \mathfrak{I_2})(U)=(\mathfrak{I_1} \cap \mathfrak{I_2})(X)=(\mathfrak{I_1} \cap \mathfrak{I_2})(Y)=(\mathfrak{I_1} \cap \mathfrak{I_2})(Z)=0$\\

Setze nun die neue Belegung in die Formel ein:\\

$(0 \vee (\neg 0 \wedge 0) \vee 0) \vee (0 \wedge 0) \equiv (1 \wedge 0) \vee 0 \equiv 0$ \qquad Widerspruch!\\

Das $(\mathfrak{I_1} \cap \mathfrak{I_2})$ ist also kein Modell von der gegebener Formel. Dies widerspricht der Abgeschlossenheit unter Schnittbildung.\\
Die Formel ist somit nicht äquivalent zu einer Horn-Formel.\\
(iii)\\
$X \wedge \neg ( \neg Y \rightarrow (\neg Y \wedge X)) \wedge ((X \wedge Y)\rightarrow(Y \vee \neg Z))$\\
$\equiv X \wedge \neg ( Y \vee (\neg Y \wedge X)) \wedge (\neg (X \wedge Y)\vee(Y \vee \neg Z))$\\
$\equiv X \wedge \neg ( Y \vee X) \wedge (\neg (X \wedge Y)\vee(Y \vee \neg Z))$\\
$\equiv X \wedge \neg Y \wedge \neg X \wedge (\neg X \vee \neg Y)\vee(Y \vee \neg Z)$\\
$\equiv X \wedge \neg Y \wedge \neg X \wedge (\neg X \vee \neg Y \vee Y \vee \neg Z)$\\
$\equiv X \wedge \neg Y \wedge \neg X \wedge 1$\\
$\equiv X \wedge \neg Y \wedge \neg X$\\

Die Formel ist äquivalent zu einer Horn-Formel.


\end{adjustwidth}

\end{document}
