\documentclass[11pt, a4paper]{article}

\usepackage{graphicx} 
\usepackage[utf8]{inputenc}
\usepackage{fancyhdr}
\usepackage{changepage}
\usepackage[onehalfspacing]{setspace}
\usepackage{ragged2e}
\usepackage{ amssymb, amsmath, amsthm, dsfont }
\usepackage[width = 18cm, top = 2.5cm, bottom = 3cm]{geometry}
\usepackage{extarrows}
\usepackage{stmaryrd}
% ---------

\newcommand{\myTitleString} {}
\newcommand{\myAuthorString} {}
\newcommand{\mySubTitleString} {}
\newcommand{\myDateString} {}

\newcommand{\myTitle}[1] {\renewcommand {\myTitleString}{#1}}
\newcommand{\mySubTitle}[1] {\renewcommand {\mySubTitleString}{#1}}
\newcommand{\myAuthor}[1] {\renewcommand{\myAuthorString}		{#1}}
\newcommand{\myDate}[1] {\renewcommand{\myDateString}{#1}}

\newcommand{\makeMyTitle}
{
\pagestyle{fancy}
\fancyhead[L]
{
\begin{tabular}{l}
\myTitleString
\\ \mySubTitleString 
\\ \myDateString
\end{tabular}
} 			
\fancyhead[C]{}
\fancyhead[R]{\myAuthorString}
\fancyfoot[C]{\thepage}
}

\setlength{\headheight}{45pt}

\newcommand{\p}{8}
\newcommand{\pp}{5}
\newcommand{\ppp}{8}
\newcommand{\pppp}{7} 

\newcommand{\defgl}{\mathrel{=\!\!\mathop:}}
\newcommand{\defgr}{\mathrel{\mathop:\!\!=}}

\makeatletter
\renewcommand*\env@matrix[1][*\c@MaxMatrixCols c]{%
  \hskip -\arraycolsep
  \let\@ifnextchar\new@ifnextchar
  \array{#1}}
\makeatother
% ---------
%\setlength{\parindent}{0pt}
\begin{document}

\myTitle{\textsc{Mathematische Logik}}
\mySubTitle{Übung 3}
\myDate{24. April 2017}
\myAuthor
{
\begin{tabular}{l l}
346532, &Daniel Boschmann\\
348776, &Anton Beliankou	\\
356092, &Daniel Schleiz
\end{tabular}
}
\makeMyTitle

\begin{tabular}{|c|c|c|c|c|}\hline
   2 & 3 & 4 & 5 &$\sum$\\\hline
  	 \qquad/\p & \qquad/\pp & \qquad/\ppp & \qquad/\pppp &\qquad/28\\\hline
 \end{tabular}
\hspace*{20pt} {\huge Gruppe \textbf{G}}
\vspace*{30pt}


\section*{Aufgabe 2 (Punkte:\qquad/\p)}
\textbf{(a)}\\
\null\ \ \textbf{(i)}
\begin{adjustwidth}{20pt}{20pt}
Die Aussage ist falsch. Seien $\varphi=0, \psi=1$. Dann gilt $\varphi \rightarrow \psi \models \varphi$ nicht, da $\varphi \rightarrow \psi$ eine Tautologie ist und insbesondere
jede Interpretation dazu passt, während $\varphi$ unerfüllbar ist und ebenfalls jede Interpretation dazu passt.
\end{adjustwidth}
\null\ \ \textbf{(ii)}
\begin{adjustwidth}{20pt}{20pt}
Die Aussage ist wahr. Zeige dazu beide Richtungen der Aussage:
\end{adjustwidth}
\begin{itemize}
\item $"\Rightarrow"$:\\
Es gelte $\Phi \models \psi$. Dann gilt für alle Modelle $\mathfrak{I}$ von $\Phi$, dass $\llbracket \psi\rrbracket^{\mathfrak{I}}=1$. Da somit
$\llbracket \neg\psi\rrbracket^{\mathfrak{I}}=0$ für alle Modelle $\mathfrak{I}$ von $\Phi$ , existiert kein Modell für $\Phi \cup \{ \neg\psi \}$, also unerfüllbar.
\item $"\Leftarrow"$:\\
Sei $\Phi \cup \{ \neg\psi \}$ unerfüllbar. Betrachte zwei Fälle: Ist $\Phi$ unerfüllbar, so gilt $\Phi \models \psi$, da für alle Modelle von $\Phi$, welche nicht existieren, gilt, dass
diese auch Modell von $\psi$ sind. Ist aber $\Phi$ erfüllbar, so besitzt $\Phi$ mindestens ein Modell. Da angenommen wurde, dass $\Phi \cup \{ \neg\psi \}$ unerfüllbar ist, gilt für
alle Modelle $\mathfrak{I}$ von $\Phi$, dass $\llbracket \neg\psi\rrbracket^{\mathfrak{I}}=0$., da sonst $\Phi \cup \{ \neg\psi \}$ erfüllbar wäre. Somit gilt für diese Modelle auch
$\llbracket \psi\rrbracket^{\mathfrak{I}}=1$ und damit folgt $\Phi \models \psi$.
\end{itemize}
\null\ \ \textbf{(iii)}
\begin{adjustwidth}{20pt}{20pt}
Die Aussage ist wahr. Da $\Phi \models \psi$ für alle $\psi \in \Psi$ gilt, ist jedes (zu $\Phi \cup \Psi$ passende) Modell von $\Phi$ ebenfalls Modell von $\Psi$. Gilt nun $\Psi \models \varphi$,
so ist jedes der eben erwähnten Modelle ebenfalls Modell von $\varphi$. Also gilt auch $\Phi \models \varphi$.
\end{adjustwidth}
\textbf{(b)}
\begin{adjustwidth}{20pt}{20pt}
Mit (a)(ii) ist die Gültigkeit der gegebenen Folgerungsbeziehung äquivalent zur Unerfüllbarkeit von 
\begin{align*}
	& \{ Y \vee \neg Z \vee Q, \neg Y \vee \neg Z, U \vee Y \vee \neg Q, U \vee X, \neg X \vee Y \vee \neg Z\} \cup \{ \neg (Z \rightarrow (U \wedge Q))\}\\
	\equiv & \{ Y \vee \neg Z \vee Q, \neg Y \vee \neg Z, U \vee Y \vee \neg Q, U \vee X, \neg X \vee Y \vee \neg Z\} \cup \{ Z \wedge (\neg U \vee \neg Q)\}
\end{align*}
Überführe die Formelmenge in die Klauselmenge
\[
	K=\{ \{ Y, \neg Z, Q\}, \{ \neg Y, \neg Z\}, \{ U, Y, \neg Q\}, \{ U, X\}, \{\neg X, Y, \neg Z\}, \{ Z\}, \{ \neg U, \neg Q\} \}.
\]
Resolviere $\{ Y, \neg Z, Q\}$ mit $\{ \neg Y, \neg Z\}$ und erhalte die Resolvente $C_1\defgr\{\neg Z, Q\}$. Resolviere $C_1$ mit $\{ \neg U, \neg Q\}$ und erhalte die
Resolvente $C_2\defgr\{\neg Z, \neg U\}$. Resolviere $C_2$ mit $\{ U, X\}$, erhalte die Resolvente $C_3\defgr \{\neg Z, X\}$. Resolviere $C_3$ mit $\{\neg X, Y, \neg Z\}$ und
erhalte die Resolvente $C_4\defgr \{\neg Z, Y\}$. Resolviere $C_4$ mit $\{ \neg Y, \neg Z\}$ und erhalte die Resolvente $C_5\defgr \{ \neg Z\}$. Resolviere nun noch $C_5 = \{ \neg Z\}$
mit $\{ Z\}$ und erhalte schließlich die leere Klausel $\square$.\\
Da die leere Klausel ableitbar ist, ist die Klauselmenge $K$ unerfüllbar. Somit folgt die Gültgkeit der Folgerungsbeziehung.
\end{adjustwidth}


\section*{Aufgabe 3 (Punkte:\qquad/\pp)}
\begin{adjustwidth}{20pt}{20pt}
Beobachte, dass bei einer Resolution von zwei Klauseln mit höchstens 2 Literalen die Resolvente wieder höchstens zwei Literale enthält. Da in den beiden Klauseln das gleiche Literal einmal
negativ und einmal positiv auftreten muss, damit eine Resolvente existiert, können in der Resolvente höchstens zwei Literale enthalten sein, falls die jeweils anderen Literale der Klauseln ungleich waren, ansonsten enthält die Resolvente nur ein Literal.\par
In einer gegebenen Klauselmenge mit $n$ Variablen treten höchstens $2n$ Literale auf (positiv und negativ). Es existieren maximal
\[
	\binom{2n}{2}+\binom{2n}{1}+\binom{2n}{0} = \frac{2n(2n-1)}{2}+2n+1 = 2n^2+n+1 \in \mathcal{O}(n^2)
\]
Klauseln mit höchstens 2 Literalen. Mithilfe der Resolution bricht man in maximal so vielen Schritten ab, das heißt das Problem ist mithilfe der Resolution in polynomieller Zeit lösbar.\par
Für Klauselmengen mit höchstens 3 Literalen klappt diese Überlegung nicht, da bei einem Resolutionsschritt eine Resolvente mit 4 Literalen entstehen kann, danach könnten immer
größere Resolventen entstehen. Die Anzahl der möglichen verschiedenen Klauseln wäre wieder nur durch $2^{2n}$ beschränkt, d.h. im Worst-Case exponentielle Laufzeit.
\end{adjustwidth}





\section*{Aufgabe 4 (Punkte:\qquad/\ppp)}
\textbf{(a)}
\begin{adjustwidth}{20pt}{20pt}
Definiere $f:\{ 1,2,3\} \to \{ a,b\}$ mit $f(1)=a, f(2)=b, f(3)=a$. Für die Kante $(1,2)$ in $H$ ist $(f(1),f(2))=(a,b)$ in $G$, für die Kante $(2,3)$ in $H$ liegt $(f(2),f(3))=(b,a)$ in $G$.
Aufgrund der Existenz der Abbildung $f$ ist $H$ homomorph zu $G$.
\end{adjustwidth}
\textbf{(b)}
\begin{adjustwidth}{20pt}{20pt}
Seien ein endlicher Graph $H=(V', E')$ und ein Graph $G=(V,E)$. Zeige, dass $G$ genau dann homomorph zu $H$ ist, wenn jeder endliche Untergraph von $G$ homomorph zu $H$ ist. \par
Konstruiere zunächst eine aussagenlogische Formel $\Phi$, welche erfüllbar ist gdw. $G$ homomorph zu $H$ ist. Definiere dazu:
\begin{align*}
	\Phi_1 & \defgr \{ \bigvee_{v'\in V'} X_{v\mapsto v'}\ |\ v\in V \}\\
	\Phi_2 & \defgr \{ X_{uv}\ |\ (u,v)\in E'\}\\
	\Phi_3 & \defgr \{ \neg X_{uv}\ |\ u,v\in V', (u,v)\notin E'\}\\
	\Phi_4 & \defgr \{ (X_{v\mapsto v'} \wedge X_{w\mapsto w'}) \rightarrow X_{v'w'}\ |\ (v,w) \in E, v', w' \in V'\}\\
	\Phi     & \defgr \Phi_1 \cup \Phi_2 \cup \Phi_3 \cup \Phi_4
\end{align*}
Die erste Formelmenge $\Phi_1$ stellt sicher, dass jeder Knoten $v \in V$ auf einen Knoten $v'$ in $H$ abgebildet wird, eine 1 bei $X_{v\mapsto v'}$ soll dies symbolisieren.
Die Disjunktion erlaubt, dass theoretisch $v$ auf mehrere Knoten abgebildet werden könnte, jedoch erschwert dies nur $\Phi_4$ und ändert nichts an der eigentlichen Aussage.\\
Die Formelmengen $\Phi_2$ und $\Phi_3$ symbolisieren die Kanten in $H$, wobei für eine erfüllende Belegung $X_{uv}=1$ gewählt werden muss, falls $(u,v)\in E'$, und 0 falls
die Kante in $H$ nicht existiert.\\
$\Phi_4$ prüft nun das eigentliche Homomorphiekriterium: Für jede Kante in $G$ existiert für alle möglichen Abbildungen von dessen Endknoten auf Knoten in $H$
($X_{v\mapsto v'} \wedge X_{w\mapsto w'}$) eine Formel. Sind dies nicht die in $\Phi_1$ "gewählten" Abbildungen, d.h. mindestens eine der Variablen ist 0, so ist die Formel
automatisch erfüllt, die linke Seite der Implikation falsch ist. Sind aber nun die in $\Phi_1$ auf 1 gesetzten Variablen enthalten, so muss die Kante der abgebildeten Knoten in
$H$ existieren (rechte Seite der Implikation), da sonst die Formel nicht erfüllt wäre. Es ist nicht schlimm, wenn ein Modell einen Knoten in $\Phi_1$ auf mehrere Knoten abbildet (mehrere
Variablen innerhalb einer der Disjunktionen werden auf 1 gesetzt), denn durch Streichen überflüssiger 1en im Modell ändert sich nichts am Wahrheitswert von $\Phi$, weil
so nur ein paar Implikationen in $\Phi_4$ trivialerweise durch eine 0 auf der linken Seite erfüllt werden.\\
Also liefert jedes Modell von $\Phi$ einen korrekten Homomorphismus, und umgekehrt. (Wähle die Kantenvariablen in $\Phi_2$ und $\Phi_3$ wie bereits erläutert entsprechend
der Kanten in $H$ und $X_{v\mapsto v'}$ gleich 1, wenn der korrekte Homomorphismus $v$ auf $v'$ abbildet.)\\ \ \\
Die Richtung, dass $G$ homomorph zu $H$ ist, wenn jeder endliche Untergraph von $G$ homomorph zu $H$ ist, ist klar, da sich der Homomorphismus auf die Knoten eines jeden
Untergraphen einschränken lässt und die geforderte Bedingung immer noch gilt, weil höchstens Kanten im Untergraphen wegfallen und $H$ gleich bleibt.\\
Bleibt also noch zu zeigen, dass wenn jeder endliche Untergraph von $G$ homomorph zu $H$ ist, dann $G$ homomorph zu $H$, wobei $G$ homomorph zu $H$ äquivalent zur Aussage
ist, dass $\Phi$ erfüllbar. Dies ist wiederum mit dem Kompaktheitssatz äquivalent zur Aussage, dass jede endliche Teilmenge $\Phi_0 \subseteq \Phi$ erfüllbar. Sei also jeder endliche
Untergraph von $G$ homomorph zu $H$ und $\Phi_0 \subseteq \Phi$ eine beliebige endliche Teilmenge. Bezeichne $G_0$ den Untergraphen von $G$, induziert von denjenigen Knoten
$v \in V$, sodass eine Variable $X_{v\mapsto v'}$ in $\Phi_0$ vorkommt. $G_0$ ist endlich, weshalb nach Annahme $G_0$ homomorph zu $H$ ist, also existiert ein solcher Homomorphismus h.
Sei nun die Interpretation $\mathfrak{I_0}$ mit 
\[
	\mathfrak{I_0}(X_{uv})=
		\begin{cases}
		1, &\quad \text{falls } (u,v) \in E\\
		0, &\quad \text{sonst}
		\end{cases}
	\qquad \text{und} \qquad
	\mathfrak{I_0}(X_{v\mapsto v'})=
		\begin{cases}
		1, &\quad \text{falls } h(v)=v'\\
		0, &\quad \text{sonst}
		\end{cases}
\]
$\mathfrak{I_0}$ ist dabei Modell von $\Phi_0$ aufgrund vorangegangener Überlegungen. Somit gilt die zu zeigende Aussage.

\end{adjustwidth}


\section*{Aufgabe 5 (Punkte:\qquad/\pppp)}
\begin{adjustwidth}{20pt}{20pt}
	Definiere wie in der Aufgabestellung vorgeschlagen für jede Teilmenge $u \in \mathbb{N}$ eine Aussagenvariable $X_u$ mit der intendierten Interpretation, dass $X_u$ wahr sein soll,
	wenn $u$ groß ist.Definiere für die Bedingungen (1)-(5) die Formelmengen 
	\begin{align*}
		\Phi_1 & \defgr \{ X_u \vee \neg X_u\ |\ u \subseteq \mathbb{N}\}\\
		\Phi_2 & \defgr \{ X_u \rightarrow \neg X_{\mathbb{N} \setminus u}\ |\ u \subseteq \mathbb{N}\}\\
		\Phi_3 & \defgr \{ \neg X_u \rightarrow \neg X_v\ |\ v \subseteq u \subseteq \mathbb{N}\}\\
		\Phi_4 & \defgr \{ (\neg X_u \wedge \neg X_v) \rightarrow \neg X_w\ |\ u,v,w \subseteq \mathbb{N}, w=u \cup v\}\\
		\Phi_5 & \defgr \{ \neg X_u\ |\ u \subseteq \mathbb{N} \text{ ist endlich}\}\\
		\Phi     & \defgr \Phi_1 \cup \Phi_2 \cup \Phi_3 \cup \Phi_4 \cup \Phi_5
\end{align*}
Nun ist $\Phi$ erfüllt genau dann, wenn eine solche Einteilung in große und kleine Mengen existiert. Zeige also, dass $\Phi$ erfüllbar ist. Mit dem Kompaktheitssatz ist dies
äquivalent zur Aussage, dass jede endliche Teilmenge von $\Phi$ erfüllbar ist, zeige dies nun.\\
Sei eine beliebige, endliche Teilmenge $\Phi_0 \subseteq \Phi$. Dann existiert eine endliche Menge $N \defgr \{ u \subseteq \mathbb{N}\ |\ X_u \in \Phi_0\}$.Die Bedingungen
(1)-(4) sind für $N$ erfüllt, da man ein festes $n \in \mathbb{N}$ wählen kann und eine Menge groß ist, genau dann wenn sie $n$ enthält. (1) ist erfüllt, da eine Menge entweder
$n$ enthält oder nicht und somit entweder groß oder klein ist. (2) ist erfüllt, denn eine Menge enthält $n$ genau dann, wenn das Komplement $n$ nicht enthält. Eine Teilmenge
einer kleinen Menge (welche $n$ nicht enthält), kann ebenfalls nicht $n$ enthalten, sonst wäre es keine Teilmenge (3). Die Vereinigung von zwei Mengen, die $n$ nicht enthalten
(und somit klein sind), kann ebenfalls $n$ nicht enthalten, somit gilt (4). Da nun noch $N$ endlich ist, wird (5) von (1)-(4) impliziert, also existiert ein Modell für $\Phi_0$,
d.h. $\Phi_0$ ist erfüllbar. Da $\Phi_0$ beliebig aber endlich gewählt war, ist mit dem Kompaktheitssatz auch $\Phi$ erfüllbar. Somit ist die Aussage, dass man die Teilmengen von
$\mathbb{N}$ in große und kleine Mengen einteilen kann, gezeigt.
\end{adjustwidth}



\end{document}
