\documentclass[11pt, a4paper]{article}

\usepackage{graphicx} 
\usepackage[utf8]{inputenc}
\usepackage{fancyhdr}
\usepackage{changepage}
\usepackage[onehalfspacing]{setspace}
\usepackage{ragged2e}
\usepackage{ amssymb, amsmath, amsthm, dsfont }
\usepackage[width = 18cm, top = 2.5cm, bottom = 3cm]{geometry}
\usepackage{extarrows}
\usepackage{stmaryrd}
% ---------

\newcommand{\myTitleString} {}
\newcommand{\myAuthorString} {}
\newcommand{\mySubTitleString} {}
\newcommand{\myDateString} {}

\newcommand{\myTitle}[1] {\renewcommand {\myTitleString}{#1}}
\newcommand{\mySubTitle}[1] {\renewcommand {\mySubTitleString}{#1}}
\newcommand{\myAuthor}[1] {\renewcommand{\myAuthorString}		{#1}}
\newcommand{\myDate}[1] {\renewcommand{\myDateString}{#1}}

\newcommand{\makeMyTitle}
{
\pagestyle{fancy}
\fancyhead[L]
{
\begin{tabular}{l}
\myTitleString
\\ \mySubTitleString 
\\ \myDateString
\end{tabular}
} 			
\fancyhead[C]{}
\fancyhead[R]{\myAuthorString}
\fancyfoot[C]{\thepage}
}

\setlength{\headheight}{45pt}

\newcommand{\p}{8}
\newcommand{\pp}{5}
\newcommand{\ppp}{8}
\newcommand{\pppp}{7} 

\newcommand{\defgl}{\mathrel{=\!\!\mathop:}}
\newcommand{\defgr}{\mathrel{\mathop:\!\!=}}

\makeatletter
\renewcommand*\env@matrix[1][*\c@MaxMatrixCols c]{%
  \hskip -\arraycolsep
  \let\@ifnextchar\new@ifnextchar
  \array{#1}}
\makeatother
% ---------
%\setlength{\parindent}{0pt}
\begin{document}

\myTitle{Mathematische Logik}
\mySubTitle{Übung X}
\myDate{24. April 2017}
\myAuthor
{
\begin{tabular}{l l}
346532, &Daniel Boschmann\\
348776, &Anton Beliankou	\\
356092, &Daniel Schleiz
\end{tabular}
}
\makeMyTitle

\begin{tabular}{|c|c|c|c|c|}\hline
   2 & 3 & 4 & 5 &$\sum$\\\hline
  	 \qquad/\p & \qquad/\pp & \qquad/\ppp & \qquad/\pppp &\qquad/28\\\hline
 \end{tabular}
\hspace*{20pt} {\huge Gruppe \textbf{G}}
\vspace*{30pt}


\section*{Aufgabe 2 (Punkte:\qquad/\p)}
\textbf{(a)}\\
\null\ \ \textbf{(i)}
\begin{adjustwidth}{20pt}{20pt}
Die Aussage ist falsch. Seien $\varphi=0, \psi=1$. Dann gilt $\varphi \rightarrow \psi \models \varphi$ nicht, da $\varphi \rightarrow \psi$ eine Tautologie ist und insbesondere
jede Interpretation dazu passt, während $\varphi$ unerfüllbar ist und ebenfalls jede Interpretation dazu passt.
\end{adjustwidth}
\null\ \ \textbf{(ii)}
\begin{adjustwidth}{20pt}{20pt}
Die Aussage ist wahr. Zeige dazu beide Richtungen der Aussage:
\end{adjustwidth}
\begin{itemize}
\item $"\Rightarrow"$:\\
Es gelte $\Phi \models \psi$. Dann gilt für alle Modelle $\mathfrak{I}$ von $\Phi$, dass $\llbracket \psi\rrbracket^{\mathfrak{I}}=1$. Da somit
$\llbracket \neg\psi\rrbracket^{\mathfrak{I}}=0$ für alle Modelle $\mathfrak{I}$ von $\Phi$ , existiert kein Modell für $\Phi \cup \{ \neg\psi \}$, also unerfüllbar.
\item $"\Leftarrow"$:\\
Sei $\Phi \cup \{ \neg\psi \}$ unerfüllbar. Betrachte zwei Fälle: Ist $\Phi$ unerfüllbar, so gilt $\Phi \models \psi$, da für alle Modelle von $\Phi$, welche nicht existieren, gilt, dass
diese auch Modell von $\psi$ sind. Ist aber $\Phi$ erfüllbar, so besitzt $\Phi$ mindestens ein Modell. Da angenommen wurde, dass $\Phi \cup \{ \neg\psi \}$ unerfüllbar ist, gilt für
alle Modelle $\mathfrak{I}$ von $\Phi$, dass $\llbracket \neg\psi\rrbracket^{\mathfrak{I}}=0$., da sonst $\Phi \cup \{ \neg\psi \}$ erfüllbar wäre. Somit gilt für diese Modelle auch
$\llbracket \psi\rrbracket^{\mathfrak{I}}=1$ und damit folgt $\Phi \models \psi$.
\end{itemize}
\null\ \ \textbf{(iii)}
\begin{adjustwidth}{20pt}{20pt}
Die Aussage ist wahr. Da $\Phi \models \psi$ für alle $\psi \in \Psi$ gilt, ist jedes (zu $\Phi \cup \Psi$ passende) Modell von $\Phi$ ebenfalls Modell von $\Psi$. Gilt nun $\Psi \models \varphi$,
so ist jedes der eben erwähnten Modelle ebenfalls Modell von $\varphi$. Also gilt auch $\Phi \models \varphi$.
\end{adjustwidth}
\textbf{(b)}
\begin{adjustwidth}{20pt}{20pt}
Mit (a)(ii) ist die Gültigkeit der gegebenen Folgerungsbeziehung äquivalent zur Unerfüllbarkeit von 
\begin{align*}
	& \{ Y \vee \neg Z \vee Q, \neg Y \vee \neg Z, U \vee Y \vee \neg Q, U \vee X, \neg X \vee Y \vee \neg Z\} \cup \{ \neg (Z \rightarrow (U \wedge Q))\}\\
	\equiv & \{ Y \vee \neg Z \vee Q, \neg Y \vee \neg Z, U \vee Y \vee \neg Q, U \vee X, \neg X \vee Y \vee \neg Z\} \cup \{ Z \wedge (\neg U \vee \neg Q)\}
\end{align*}
Überführe die Formelmenge in die Klauselmenge
\[
	K=\{ \{ Y, \neg Z, Q\}, \{ \neg Y, \neg Z\}, \{ U, Y, \neg Q\}, \{ U, X\}, \{\neg X, Y, \neg Z\}, \{ Z\}, \{ \neg U, \neg Q\} \}.
\]
Resolviere $\{ Y, \neg Z, Q\}$ mit $\{ \neg Y, \neg Z\}$ und erhalte die Resolvente $C_1\defgr\{\neg Z, Q\}$. Resolviere $C_1$ mit $\{ \neg U, \neg Q\}$ und erhalte die
Resolvente $C_2\defgr\{\neg Z, \neg U\}$. Resolviere $C_2$ mit $\{ U, X\}$, erhalte die Resolvente $C_3\defgr \{\neg Z, X\}$. Resolviere $C_3$ mit $\{\neg X, Y, \neg Z\}$ und
erhalte die Resolvente $C_4\defgr \{\neg Z, Y\}$. Resolviere $C_4$ mit $\{ \neg Y, \neg Z\}$ und erhalte die Resolvente $C_5\defgr \{ \neg Z\}$. Resolviere nun noch $C_5 = \{ \neg Z\}$
mit $\{ Z\}$ und erhalte schließlich die leere Klausel $\square$.\\
Da die leere Klausel ableitbar ist, ist die Klauselmenge $K$ unerfüllbar. Somit folgt die Gültgkeit der Folgerungsbeziehung.
\end{adjustwidth}
\section*{Aufgabe 3 (Punkte:\qquad/\pp)}

\begin{adjustwidth}{20pt}{20pt}

\end{adjustwidth}





\section*{Aufgabe 4 (Punkte:\qquad/\ppp)}
\textbf{(a)}
\begin{adjustwidth}{20pt}{20pt}


\end{adjustwidth}
\textbf{(b)}
\begin{adjustwidth}{20pt}{20pt}


\end{adjustwidth}


\section*{Aufgabe 5 (Punkte:\qquad/\pppp)}

\begin{adjustwidth}{20pt}{20pt}

\end{adjustwidth}



\end{document}
