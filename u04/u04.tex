\documentclass[11pt, a4paper]{article}

\usepackage{graphicx}
\usepackage{forest}
\usepackage[utf8]{inputenc}
\usepackage{fancyhdr}
\usepackage{changepage}
\usepackage[onehalfspacing]{setspace}
\usepackage{ragged2e}
\usepackage{ amssymb, amsmath, amsthm, dsfont }
\usepackage[width = 18cm, top = 2.5cm, bottom = 3cm]{geometry}
\usepackage{extarrows}
\usepackage{stmaryrd}
% ---------

\newcommand{\myTitleString} {}
\newcommand{\myAuthorString} {}
\newcommand{\mySubTitleString} {}
\newcommand{\myDateString} {}

\newcommand{\myTitle}[1] {\renewcommand {\myTitleString}{#1}}
\newcommand{\mySubTitle}[1] {\renewcommand {\mySubTitleString}{#1}}
\newcommand{\myAuthor}[1] {\renewcommand{\myAuthorString}		{#1}}
\newcommand{\myDate}[1] {\renewcommand{\myDateString}{#1}}

\newcommand{\makeMyTitle}
{
\pagestyle{fancy}
\fancyhead[L]
{
\begin{tabular}{l}
\myTitleString
\\ \mySubTitleString 
\\ \myDateString
\end{tabular}
} 			
\fancyhead[C]{}
\fancyhead[R]{\myAuthorString}
\fancyfoot[C]{\thepage}
}

\setlength{\headheight}{45pt}

\newcommand{\p}{9}
\newcommand{\pp}{4}
\newcommand{\ppp}{9}
\newcommand{\pppp}{8} 

\newcommand{\defgl}{\mathrel{=\!\!\mathop:}}
\newcommand{\defgr}{\mathrel{\mathop:\!\!=}}

\makeatletter
\renewcommand*\env@matrix[1][*\c@MaxMatrixCols c]{%
  \hskip -\arraycolsep
  \let\@ifnextchar\new@ifnextchar
  \array{#1}}
\makeatother
% ---------
%\setlength{\parindent}{0pt}
\begin{document}

\myTitle{\textsc{Mathematische Logik}}
\mySubTitle{Übung 4}
\myDate{13. Mai 2017}
\myAuthor
{
\begin{tabular}{l l}
346532, &Daniel Boschmann\\
348776, &Anton Beliankou	\\
356092, &Daniel Schleiz
\end{tabular}
}
\makeMyTitle

\begin{tabular}{|c|c|c|c|c|}\hline
   2 & 3 & 4 & 5 &$\sum$\\\hline
  	 \qquad/\p & \qquad/\pp & \qquad/\ppp & \qquad/\pppp &\qquad/30\\\hline
 \end{tabular}
\hspace*{20pt} {\huge Gruppe \textbf{G}}
\vspace*{30pt}


\section*{Aufgabe 2 (Punkte:\qquad/\p)}
\textbf{(a)}
\begin{adjustwidth}{20pt}{20pt}
	Sei eine Klauselmenge $K$ ohne positive Klauseln gegeben und seien $\{ X_1,...,X_n\}$ die darin vorkommenden Aussagenvariablen. Da $K$ keine
	positiven Klauseln enthält, folgt, dass jede Klausel mindestens ein negatives Literal enthält. Es ist $\mathfrak{I}$ mit $\mathfrak{I}(X_i)=0$ für $i \in \{ 1,...,n\}$
	ein Modell jeder Klausel in $K$, da in einer Klausel stets ein negatives Literal auftritt. Somit folgt, dass auch $K$ durch $\mathfrak{I}$ erfüllt wird. Da $K$ eine
	beliebige Klauselmenge ohne positive Klauseln ist, gilt die Aussage.
\end{adjustwidth}
\textbf{(b)}
\begin{adjustwidth}{20pt}{20pt}
\begin{forest}
  for tree={
    grow'=90,
    parent anchor=north,
    math content,
    before typesetting nodes={
      if level=0{}{
        if content={}{
          shape=coordinate
        }{
          content/.wrap value={\{#1\}},
        },
      },
    }
  }
  [\Box
    [{Y}
      [{Y,\neg X}
        [{Z,Y}
          [{\neg X,Z,Y}]
          [{X,Y}]
        ]
        [
          [{\neg X,\neg Z}]
        ]
      ]
      [
        [
          [{X,Y}]
        ]
      ]
    ]
    [
      [
        [
          [{\neg Y}]
        ]
      ]
    ]
  ]
\end{forest}\ \\
Da die leere Klausel mittels P-Resolution ableitbar ist, ist $K$ unerfüllbar.
\end{adjustwidth}
\textbf{(c)}
\begin{adjustwidth}{20pt}{20pt}

\end{adjustwidth}
\textbf{(d)}
\begin{adjustwidth}{20pt}{20pt}

\end{adjustwidth}



\section*{Aufgabe 3 (Punkte:\qquad/\pp)}
\textbf{(a)}
\begin{adjustwidth}{20pt}{20pt}
Die Sequenz ist nicht gültig. Betrachte hierzu die Interpretation $\mathfrak{I}:U \mapsto 1,Z \mapsto 0, Y \mapsto 1, X \mapsto 0$, welche beide Formeln auf der rechten Seite
der Implikation nicht erfüllt. Da diese Interpretation erfüllend ist für beide Formeln auf der linken Seite der Implikation, ist die Sequenz ungültig, da eine falsifizierende Interpretation
existiert.
\end{adjustwidth}
\textbf{(b)}
\begin{adjustwidth}{20pt}{20pt}
Angenommen, es existiert eine falsifizierende Interpretation $\mathfrak{I}$ für die gegebene Sequenz, d.h. $\mathfrak{I}$ erfüllt alle Formeln auf der linken Seite der Interpretation
und keine auf der rechten Seite. Es muss $\mathfrak{I}(X)=1$ gelten, da $\mathfrak{I}$ die linke Seite erfüllen muss und $X$ in beiden Konjunktionen innerhalb der Klammern auftritt.
Da $\mathfrak{I}$ die rechte Seite nicht erfüllen darf, muss $\mathfrak{I}(U)=0$ gelten, da sonst $U \wedge X$ erfüllt wäre. Da nun $\llbracket (X \wedge U)\rrbracket^\mathfrak{I}=0$,
muss $\mathfrak{I}$ die linke Konjunktion, $(X \wedge \neg Y \wedge Z)$, erfüllen, das heißt es müsste $\mathfrak{I}(Y)=0$ und $\mathfrak{I}(Z)=1$ gelten. Dann ist aber
$\llbracket (\neg Y \wedge \neg U)\rrbracket^\mathfrak{I}=1$, und somit ist eine Formel der rechten Seite erfüllt. Dies steht im Widerspruch zur Annahme, dass $\mathfrak{I}$ eine
falsifizierende Interpretation ist. $\lightning$\\
Da also keine falsifizierende Interpretation für die gegebene Sequenz existiert, ist die Sequenz gültig.
\end{adjustwidth}



\section*{Aufgabe 4 (Punkte:\qquad/\ppp)}
\textbf{(a)}
\begin{adjustwidth}{20pt}{20pt}

\end{adjustwidth}
\textbf{(b)}
\begin{adjustwidth}{20pt}{20pt}

\end{adjustwidth}



\section*{Aufgabe 5 (Punkte:\qquad/\pppp)}
\textbf{(a)}
\begin{adjustwidth}{20pt}{20pt}

\end{adjustwidth}
\textbf{(b)}
\begin{adjustwidth}{20pt}{20pt}

\end{adjustwidth}
\textbf{(c)}
\begin{adjustwidth}{20pt}{20pt}

\end{adjustwidth}



\end{document}
