\documentclass[11pt, a4paper]{article}

\usepackage{graphicx}
\usepackage{forest}
\usepackage{proof}
\usepackage[utf8]{inputenc}
\usepackage{fancyhdr}
\usepackage{changepage}
\usepackage[onehalfspacing]{setspace}
\usepackage{ragged2e}
\usepackage{ amssymb, amsmath, amsthm, dsfont }
\usepackage[width = 18cm, top = 2.5cm, bottom = 3cm]{geometry}
\usepackage{extarrows}
\usepackage{stmaryrd}
% ---------

\newcommand{\myTitleString} {}
\newcommand{\myAuthorString} {}
\newcommand{\mySubTitleString} {}
\newcommand{\myDateString} {}

\newcommand{\myTitle}[1] {\renewcommand {\myTitleString}{#1}}
\newcommand{\mySubTitle}[1] {\renewcommand {\mySubTitleString}{#1}}
\newcommand{\myAuthor}[1] {\renewcommand{\myAuthorString}		{#1}}
\newcommand{\myDate}[1] {\renewcommand{\myDateString}{#1}}

\newcommand{\makeMyTitle}
{
\pagestyle{fancy}
\fancyhead[L]
{
\begin{tabular}{l}
\myTitleString
\\ \mySubTitleString 
\\ \myDateString
\end{tabular}
} 			
\fancyhead[C]{}
\fancyhead[R]{\myAuthorString}
\fancyfoot[C]{\thepage}
}

\setlength{\headheight}{45pt}

\newcommand{\p}{9}
\newcommand{\pp}{4}
\newcommand{\ppp}{9}
\newcommand{\pppp}{8} 

\newcommand{\defgl}{\mathrel{=\!\!\mathop:}}
\newcommand{\defgr}{\mathrel{\mathop:\!\!=}}

\makeatletter
\renewcommand*\env@matrix[1][*\c@MaxMatrixCols c]{%
  \hskip -\arraycolsep
  \let\@ifnextchar\new@ifnextchar
  \array{#1}}
\makeatother
% ---------
%\setlength{\parindent}{0pt}
\begin{document}

\myTitle{\textsc{Mathematische Logik}}
\mySubTitle{Übung 4}
\myDate{13. Mai 2017}
\myAuthor
{
\begin{tabular}{l l}
346532, &Daniel Boschmann\\
348776, &Anton Beliankou	\\
356092, &Daniel Schleiz
\end{tabular}
}
\makeMyTitle

\begin{tabular}{|c|c|c|c|c|}\hline
   2 & 3 & 4 & 5 &$\sum$\\\hline
  	 \qquad/\p & \qquad/\pp & \qquad/\ppp & \qquad/\pppp &\qquad/30\\\hline
 \end{tabular}
\hspace*{20pt} {\huge Gruppe \textbf{G}}
\vspace*{30pt}


\section*{Aufgabe 2 (Punkte:\qquad/\p)}
\textbf{(a)}
\begin{adjustwidth}{20pt}{20pt}
	Sei eine Klauselmenge $K$ ohne positive Klauseln gegeben und seien $\{ X_1,...,X_n\}$ die darin vorkommenden Aussagenvariablen. Da $K$ keine
	positiven Klauseln enthält, folgt, dass jede Klausel mindestens ein negatives Literal enthält. Es ist $\mathfrak{I}$ mit $\mathfrak{I}(X_i)=0$ für $i \in \{ 1,...,n\}$
	ein Modell jeder Klausel in $K$, da in einer Klausel stets ein negatives Literal auftritt. Somit folgt, dass auch $K$ durch $\mathfrak{I}$ erfüllt wird. Da $K$ eine
	beliebige Klauselmenge ohne positive Klauseln ist, gilt die Aussage.
\end{adjustwidth}
\textbf{(b)}
\begin{adjustwidth}{20pt}{20pt}
\begin{forest}
  for tree={
    grow'=90,
    parent anchor=north,
    math content,
    before typesetting nodes={
      if level=0{}{
        if content={}{
          shape=coordinate
        }{
          content/.wrap value={\{#1\}},
        },
      },
    }
  }
  [\Box
    [{Y}
      [{Y,\neg X}
        [{Z,Y}
          [{\neg X,Z,Y}]
          [{X,Y}]
        ]
        [
          [{\neg X,\neg Z}]
        ]
      ]
      [
        [
          [{X,Y}]
        ]
      ]
    ]
    [
      [
        [
          [{\neg Y}]
        ]
      ]
    ]
  ]
\end{forest}\ \\
Da die leere Klausel mittels P-Resolution ableitbar ist, ist $K$ unerfüllbar.
\end{adjustwidth}
\textbf{(c)}
\begin{adjustwidth}{20pt}{20pt}
Das P-Resolutionskalkül ist korrekt, da das Resolutionskalkül korrekt ist und die bei einer P-Resolution entstehenden Resolventen ebenfalls in der Resolution gebildet werden können.
\end{adjustwidth}
\textbf{(d)}
\begin{adjustwidth}{20pt}{20pt}
Der Beweis der Vollständigkeit der P-Resolution funktioniert wie der der Resolution bis
einschließlich der Definition von $K_0^+$ und $K_0^-$. Nach Induktionsvoraussetzung lässt sich
annehmen, dass $\Box \in \text{Res}^*(K_0^+)$ und $\Box \in \text{Res}^*(K_0^-)$, welche mittels P-Resolution abgeleitet wurden.. Betrachte nun
$K_0^-$. Da aus $K_0^-$ alle Klauseln mit $\neg X_n$ entfernt wurden und aus den verbleibenden Klauseln alle $X_n$ gestrichen wurden, lässt sich zu allen Klauseln aus $K_0^-$ wieder
$X_n$ hinzufügen, und dann ${X_n}$ mit P-Resolution ableiten, da ja $\Box$ ableitbar und durch hinzufügen von ${X_n}$ keine positive Klauseln negativ werden. Resolviere nun alle Klauseln
in $K_0$, welche $\neg X_n$ enthalten, mit ${X_n}$.(Es handelt sich um P-Resolution, da ${X_n}$ positiv ist.) Die dabei entstehende Menge ist eine Obermenge von $K_0^+$, da in den Resolventen
nun keine $\neg X_n$ enthalten sind.(Die überflüssigen Klauseln, welche in $K_0^+$ gestrichen wurden lassen sich nun ignorieren.) Nach Induktionsvoraussetzung lässt sich aber aus $K_0^+$
ebenfalls die leere Klausel mit P-Resolution ableiten, und somit insbesondere auch aus der Obermenge. Der Rest des Beweises ist dann wieder analog zur Resolution. Somit ist die
P-Resolution vollständig.
\end{adjustwidth}



\section*{Aufgabe 3 (Punkte:\qquad/\pp)}
\textbf{(a)}
\begin{adjustwidth}{20pt}{20pt}
Die Sequenz ist nicht gültig. Betrachte hierzu die Interpretation $\mathfrak{I}:U \mapsto 1,Z \mapsto 0, Y \mapsto 1, X \mapsto 0$, welche beide Formeln auf der rechten Seite
der Implikation nicht erfüllt. Da diese Interpretation erfüllend ist für beide Formeln auf der linken Seite der Implikation, ist die Sequenz ungültig, da eine falsifizierende Interpretation
existiert.
\end{adjustwidth}
\textbf{(b)}
\begin{adjustwidth}{20pt}{20pt}
Angenommen, es existiert eine falsifizierende Interpretation $\mathfrak{I}$ für die gegebene Sequenz, d.h. $\mathfrak{I}$ erfüllt alle Formeln auf der linken Seite der Interpretation
und keine auf der rechten Seite. Es muss $\mathfrak{I}(X)=1$ gelten, da $\mathfrak{I}$ die linke Seite erfüllen muss und $X$ in beiden Konjunktionen innerhalb der Klammern auftritt.
Da $\mathfrak{I}$ die rechte Seite nicht erfüllen darf, muss $\mathfrak{I}(U)=0$ gelten, da sonst $U \wedge X$ erfüllt wäre. Da nun $\llbracket (X \wedge U)\rrbracket^\mathfrak{I}=0$,
muss $\mathfrak{I}$ die linke Konjunktion, $(X \wedge \neg Y \wedge Z)$, erfüllen, das heißt es müsste $\mathfrak{I}(Y)=0$ und $\mathfrak{I}(Z)=1$ gelten. Dann ist aber
$\llbracket (\neg Y \wedge \neg U)\rrbracket^\mathfrak{I}=1$, und somit ist eine Formel der rechten Seite erfüllt. Dies steht im Widerspruch zur Annahme, dass $\mathfrak{I}$ eine
falsifizierende Interpretation ist. $\lightning$\\
Da also keine falsifizierende Interpretation für die gegebene Sequenz existiert, ist die Sequenz gültig.
\end{adjustwidth}



\section*{Aufgabe 4 (Punkte:\qquad/\ppp)}
\textbf{(a)}
\begin{adjustwidth}{20pt}{20pt}
\textbf{(i)}\\

Wende $(\Rightarrow,\wedge)$ an:\\

$
\dfrac
{U\vee \neg X,Z \Rightarrow Y,\neg (X \rightarrow Y) \text{\qquad} U\vee \neg X,Z \Rightarrow Y,\neg U \rightarrow Z}
{U\vee \neg X,Z \Rightarrow Y,\neg (X \rightarrow Y)\wedge (\neg U \rightarrow Z)}
$\\

Weiter mit $(\Rightarrow, \neg)$ auf das linke Blatt und $(\Rightarrow, \rightarrow)$  auf das rechte Blatt:\\ 

$
\dfrac
{\dfrac{U\vee \neg X,Z,X \rightarrow Y \Rightarrow Y}{U\vee \neg X,Z \Rightarrow Y,\neg (X \rightarrow Y)} \text{\qquad} \dfrac{U\vee \neg X,Z,\neg U \Rightarrow Y,Z}{U\vee \neg X,Z \Rightarrow Y,\neg U \rightarrow Z}}
{U\vee \neg X,Z \Rightarrow Y,\neg (X \rightarrow Y)\wedge (\neg U \rightarrow Z)}
$\\

Das rechte Blatt ist nun ein Axiom. Wende $(\vee, \Rightarrow)$ auf linkes Blatt.\\

$
\dfrac
{\dfrac{
		\dfrac
		{X,Z,U \Rightarrow Y \text{\qquad}  X,Z,\neg X \Rightarrow Y}
		{U\vee \neg X,Z,X \rightarrow Y \Rightarrow Y}}
{U\vee \neg X,Z \Rightarrow Y,\neg (X \rightarrow Y)} \text{\qquad} \dfrac
	{	\dfrac
		{\text{Axiom}}
		{U\vee \neg X,Z,\neg U \Rightarrow Y,Z}}
	{U\vee \neg X,Z \Rightarrow Y,\neg U \rightarrow Z}}
{U\vee \neg X,Z \Rightarrow Y,\neg (X \rightarrow Y)\wedge (\neg U \rightarrow Z)}
$\\

Nun sieht man, die falsifizierende Interpretation wäre $\mathfrak{I}: X \mapsto 1, Y \mapsto 0, Z \mapsto 1, U \mapsto 1$. Daher existiert es kein Beweis für die gegebene Sequenz. Die Sequenz ist nicht gültig.\\
\vspace*{35pt}

\textbf{(ii)}\\

Mit $(\vee \Rightarrow)$:\\

$
\dfrac
{\overbrace{X \wedge \neg Y \wedge Z \Rightarrow \neg Y \wedge \neg U, X \wedge U}^{\textbf{A}} \text{\qquad} \overbrace{X \wedge U\Rightarrow \neg Y \wedge \neg U, X \wedge U}^{\textbf{B}}}
{(X \wedge \neg Y \wedge Z)\vee(X \wedge U)\Rightarrow \neg Y \wedge \neg U, X \wedge U}
$\\

Teilbaum \textbf{A}:\\

Mit $(\wedge,\Rightarrow)$:\\

$
\dfrac
{X , \neg Y , Z \Rightarrow \neg Y \wedge \neg U, X \wedge U}
{X \wedge \neg Y \wedge Z \Rightarrow \neg Y \wedge \neg U, X \wedge U}
$\\

Weiter mit $(\Rightarrow, \wedge)$ und danach ebenfalls $(\Rightarrow, \wedge)$ auf das rechte Blatt. Anschließend $(\Rightarrow, \neg)$ auf das rechte Blatt:\\

$
\dfrac
{\dfrac
 {\dfrac{\text{Axiom}}{X , \neg Y , Z \Rightarrow \neg Y, X \wedge U} \text{\qquad} 
  \dfrac
  {\dfrac
  {\text{Axiom}}
  {X , \neg Y , Z \Rightarrow  \neg U, X}  \text{\qquad} 
   \dfrac
   {\dfrac
  {\text{Axiom}}
  {U, X , \neg Y , Z \Rightarrow U}}
   {X , \neg Y , Z \Rightarrow  \neg U, U}}
  {X , \neg Y , Z \Rightarrow  \neg U, X \wedge U}}
 {X , \neg Y , Z \Rightarrow \neg Y \wedge \neg U, X \wedge U}}
{X \wedge \neg Y \wedge Z \Rightarrow \neg Y \wedge \neg U, X \wedge U}
$\\

Teilbaum \textbf{B}:\\

Wende $(\wedge,\Rightarrow)$ auf die Wurzel an. Danach $(\Rightarrow, \wedge)$ und anschließend $(\Rightarrow, \wedge)$ auf alle Blätter:\\

$
\dfrac
{	\dfrac{ \dfrac
  {\text{Axiom}}
  {X , U \Rightarrow \neg Y , X} \text{\qquad} \dfrac
  {\text{Axiom}}
  {X , U \Rightarrow \neg Y , U}}
	{X , U \Rightarrow \neg Y , X \wedge U} \text{\qquad} 
	\dfrac{\dfrac
  {\text{Axiom}}
  {X , U \Rightarrow \neg U , X}\text{\qquad}\dfrac
  {\text{Axiom}}
  {X , U \Rightarrow \neg U , U}}
	{X , U \Rightarrow  \neg U, X \wedge U}}
{
\dfrac
{X , U\Rightarrow \neg Y \wedge \neg U, X \wedge U}
{X \wedge U\Rightarrow \neg Y \wedge \neg U, X \wedge U}
}
$\\

Da alle Blätter positiv sind, ist der Ableitungsbaum ein Beweis. Die Sequenz ist somit gültig.
\end{adjustwidth}
\textbf{(b)}
\begin{adjustwidth}{20pt}{20pt}
Sei $\varphi \in $ AL. Um die Erfüllbarkeit von $\varphi$ zu prüfen, kann man mithilfe des Sequenzenkalküls die Gültigkeit der Sequenz $\varphi \Rightarrow \O$ prüfen. Ist die Sequenz
ungültig, so ist $\varphi$ erfüllbar, ansonsten nicht. (Denn $\varphi \Rightarrow \O$ gültig gdw. $\varphi$ unerfüllbar.)\\
Dabei handelt es sich nicht um ein effizientes Verfahren, da bei der Anwendung einer Schlussregel aus einer Sequenz zwei zu beweisende Sequenzen entstehen können, weshalb
die Anzahl der entstehenden Sequenzen im Worst-Case exponentiell wachsen kann, wodurch die Anzahl der benötigten Schritte ebenfalls exponentiell wächst.
\end{adjustwidth}



\section*{Aufgabe 5 (Punkte:\qquad/\pppp)}
\textbf{(a)}
\begin{adjustwidth}{20pt}{20pt}
Die Schlussregel ist korrekt. Seien $\Gamma,\varphi \Rightarrow \Delta,\psi$ und $\Gamma,\psi \Rightarrow \Delta,\vartheta$ gültig, und sei $\mathfrak{I}$ ein Modell von
$\Gamma, \neg \vartheta$. (Falls $\Gamma, \neg \vartheta$ unerfüllbar, gilt die zu zeigende Aussage sofort.) Insbesondere ist $\mathfrak{I}$ Modell von $\Gamma$. Betrachte
nun zwei Fälle:
\begin{itemize}
\item $\mathfrak{I} \not\models \psi$. Dann ist $\Gamma, \neg \vartheta \Rightarrow \Delta, \psi \rightarrow \varphi$ gültig, da in dem Fall $\mathfrak{I}$ Modell von $\psi \rightarrow
	\varphi$ ist.
\item $\mathfrak{I} \models \psi$. Da angenommen, dass $\Gamma,\psi \Rightarrow \Delta,\vartheta$ gültig ist und in dem Fall $\mathfrak{I}$ Modell von $\Gamma,\psi$ ist, muss
	$\mathfrak{I}$ Modell von einer Formel $\delta \in \Delta$ oder von $\vartheta$. Da aber $\mathfrak{I}$ aber als Modell von $\Gamma, \neg \vartheta$ gewählt war,
	kann $\mathfrak{I}$ kein Modell von $\vartheta$ sein. Also ist $\mathfrak{I}$ Modell einer Formel $\delta \in \Delta$. Damit ist die Sequenz $\Gamma, \neg \vartheta
	\Rightarrow \Delta, \psi \rightarrow \varphi$ gültig.
\end{itemize}
Da also $\Gamma, \neg \vartheta \Rightarrow \Delta, \psi \rightarrow \varphi$ stets gültig ist unter den gegebenen Bedingungen, ist die Korrektheit dieser Schlussregel gezeigt.
\end{adjustwidth}
\textbf{(b)}
\begin{adjustwidth}{20pt}{20pt}
Die Schlussregel ist nicht korrekt. Seien $\Gamma=\O, \Delta=\O, \psi = 1, \varphi = 0$. Dann ist die Sequenz $(\Gamma,\psi \rightarrow \varphi \Rightarrow \Gamma)\equiv
(\O,1 \rightarrow 0 \Rightarrow \O)$ gültig, da die linke Seite unerfüllbar ist (aufgrund der Formel $1 \rightarrow 0$). Außerdem ist die Sequenz $(\neg \varphi \Rightarrow
\psi, \Delta)\equiv(1 \Rightarrow 1,\Delta)$ gültig, da jede Interpretation die linke Seite (bestehend aus 1) erfüllt und dabei stets die erste Formel der rechten Seite, ebenfalls 1,
stets erfüllt ist.\\
Mithilfe Der Schlussregel müsste dann auch die Sequenz $(\Gamma \Rightarrow \Delta)\equiv(\O \Rightarrow \O)$ gültig sein, dies ist aber nicht der Fall, da jedes Modell der linken
Seite (also jede Interpretation) nicht Modell von mindestens einer Formel der rechten Seite ist, da $\O$ keine Formeln enthält. Es folgt also, dass die Schlussregel nicht korrekt ist.
\end{adjustwidth}
\textbf{(c)}
\begin{adjustwidth}{20pt}{20pt}
Die Schlussregel ist korrekt. Seien $\Gamma \Rightarrow \Delta,\psi$ und $\Gamma \Rightarrow \Delta,\vartheta$ gültig. Falls $\Gamma$ unerfüllbar ist, so ist auch die Sequenz
$\Gamma \Rightarrow \Delta,\psi \wedge \vartheta$ gültig, da jedes Modell der linken Seite, welches nicht existiert, auch Modell einer Formel der rechten Seite ist. Sei nun
$\Gamma$ erfüllbar und sei $\mathfrak{I}$ ein beliebiges Modell von $\Gamma$. Betrachte zwei Fälle:
\begin{itemize}
\item $\mathfrak{I}$ ist Modell von mindestens einer Formel aus $\Delta$. Dies genügt bereits für die nachfolgende Argumentation.
\item $\mathfrak{I}$ ist Modell von keiner Formel aus $\Delta$. Es folgt nun, dass $\mathfrak{I} \models \psi$ und $\mathfrak{I} \models \vartheta$ gelten muss, da sonst entgegen
	der Annahme $\Gamma \Rightarrow \Delta,\psi$ und $\Gamma \Rightarrow \Delta,\vartheta$ nicht gültig wären. Damit gilt auch $\mathfrak{I} \models (\psi \wedge \vartheta)$. 
\end{itemize}
Somit ist jedes Modell von $\Gamma$ auch Modell einer Formel aus $\Delta$ oder Modell von $\psi \wedge \vartheta$. Es folgt die Gültigkeit von $\Gamma \Rightarrow \Delta,\psi \wedge
\vartheta$, dies war zu zeigen. Die Schlussregel ist damit korrekt.
\end{adjustwidth}



\end{document}
