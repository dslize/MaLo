\documentclass[11pt, a4paper]{article}

\usepackage{graphicx}
\usepackage{forest}
\usepackage{proof}
\usepackage[utf8]{inputenc}
\usepackage{fancyhdr}
\usepackage{changepage}
\usepackage[onehalfspacing]{setspace}
\usepackage{ragged2e}
\usepackage{ amssymb, amsmath, amsthm, dsfont }
\usepackage[width = 18cm, top = 2.5cm, bottom = 3cm]{geometry}
\usepackage{extarrows}
\usepackage{stmaryrd}
% ---------

\newcommand{\myTitleString} {}
\newcommand{\myAuthorString} {}
\newcommand{\mySubTitleString} {}
\newcommand{\myDateString} {}

\newcommand{\myTitle}[1] {\renewcommand {\myTitleString}{#1}}
\newcommand{\mySubTitle}[1] {\renewcommand {\mySubTitleString}{#1}}
\newcommand{\myAuthor}[1] {\renewcommand{\myAuthorString}		{#1}}
\newcommand{\myDate}[1] {\renewcommand{\myDateString}{#1}}

\newcommand{\makeMyTitle}
{
\pagestyle{fancy}
\fancyhead[L]
{
\begin{tabular}{l}
\myTitleString
\\ \mySubTitleString 
\\ \myDateString
\end{tabular}
} 			
\fancyhead[C]{}
\fancyhead[R]{\myAuthorString}
\fancyfoot[C]{\thepage}
}

\setlength{\headheight}{45pt}

\newcommand{\p}{7}
\newcommand{\pp}{5}
\newcommand{\ppp}{8}
\newcommand{\pppp}{6} 

\newcommand{\defgl}{\mathrel{=\!\!\mathop:}}
\newcommand{\defgr}{\mathrel{\mathop:\!\!=}}

\makeatletter
\renewcommand*\env@matrix[1][*\c@MaxMatrixCols c]{%
  \hskip -\arraycolsep
  \let\@ifnextchar\new@ifnextchar
  \array{#1}}
\makeatother
% ---------
%\setlength{\parindent}{0pt}
\begin{document}

\myTitle{\textsc{Mathematische Logik}}
\mySubTitle{Übung 5}
\myDate{21. Mai 2017}
\myAuthor
{
\begin{tabular}{l l}
346532, &Daniel Boschmann\\
348776, &Anton Beliankou	\\
356092, &Daniel Schleiz
\end{tabular}
}
\makeMyTitle

\begin{tabular}{|c|c|c|c|c|}\hline
   2 & 3 & 4 & 5 &$\sum$\\\hline
  	 \qquad/\p & \qquad/\pp & \qquad/\ppp & \qquad/\pppp &\qquad/26\\\hline
 \end{tabular}
\hspace*{20pt} {\huge Gruppe \textbf{G}}
\vspace*{30pt}


\section*{Aufgabe 2 (Punkte:\qquad/\p)}
\textbf{(a)}
\begin{adjustwidth}{20pt}{20pt}
	Es existieren die Redukte $(\mathbb{N},+,\cdot,<), (\mathbb{N},\cdot,<), (\mathbb{N},+,<), (\mathbb{N},+,\cdot,<), (\mathbb{N},+,\cdot), (\mathbb{N},<),
	(\mathbb{N},+), (\mathbb{N},\cdot)$ und $(\mathbb{N})$.
\end{adjustwidth}
\textbf{(b)}
\begin{adjustwidth}{20pt}{20pt}
	Für $n \subseteq \mathbb{N}$ und $n \neq \emptyset$ ist $\mathfrak{N}_n=(n,\leq)$ eine Substruktur von $\mathfrak{N}_1$, da sich jede Teilmenge der natürlichen Zahlen	
	auf die selbe Weise ordnen lässt. \\
	Sei $n \subseteq \mathbb{N}$, $n \neq \emptyset$ und sei $N_n=\{ 2^i \cdot m\ |\ m \in n, i \in \mathbb{N}\}$. Dann ist $\mathfrak{N}_{N_n}=(N_n, f)$ eine Substruktur
	von $\mathfrak{N}_2$, da man alle Elemente in $n$ um die Vielfachen mit allen Zweierpotenzen erweitern muss, da die Substruktur sonst nicht $\{ f \}$-abgeschlossen wäre.
\end{adjustwidth}
\textbf{(c)}
\begin{adjustwidth}{20pt}{20pt}
	$(\mathbb{Z}/3\mathbb{Z},+)$ besitzt die Substrukturen $(\mathbb{Z}/2\mathbb{Z},+)$ und $(\mathbb{Z}/1\mathbb{Z},+)=\{ 0 \}$, sich die Addition modulo problemlos
	einschränken lässt. (Für $(\mathbb{Z}/1\mathbb{Z},+)$ klar, da $0+0=0\in (\mathbb{Z}/1\mathbb{Z},+)$, für $(\mathbb{Z}/2\mathbb{Z},+)$ auch klar, bis auf
	$1+1=2 \equiv_2 0 \in (\mathbb{Z}/2\mathbb{Z},+)$, also doch abegschlossen.) \\
	$(\mathbb{Z}/4\mathbb{Z},+)$ besitzt die Substrukturen $(\mathbb{Z}/2\mathbb{Z},+)$ und $(\mathbb{Z}/1\mathbb{Z},+)$ (aus analogen Gründen zu vorher), nicht aber
	$(\mathbb{Z}/3\mathbb{Z},+)$. (Betrachte bei der Einschränkung des Definitionsbereichs auf $(\mathbb{Z}/3\mathbb{Z},+)$, nach Vorschrift von $(\mathbb{Z}/4\mathbb{Z},+)$:
	$2+2=0 \not\equiv_3 1$.)
\end{adjustwidth}




\section*{Aufgabe 3 (Punkte:\qquad/\pp)}
\textbf{(a)}
\begin{adjustwidth}{20pt}{20pt}
	Die Aussage des Satzes ist, dass ein Knoten $x$ existiert, welcher nicht inzident zu einer Kante ist. Die Aussage trifft nur auf $\mathcal{G}_2$, der Knoten oben links ist isoliert.
\end{adjustwidth}
\textbf{(b)}
\begin{adjustwidth}{20pt}{20pt}
	Die Aussage des Satzes ist, dass ein Knoten $x$ existiert, welcher über eine Kante mit zwei anderen, verschiedenen, Knoten $y,z$ verbunden ist. Dies trifft auf die Graphen
	 $\mathcal{G}_1$ ($x$ ist Knoten oben links) und  $\mathcal{G}_4$ ($x$ ist z.B. der Knoten unten rechts) zu. Der Satz gilt für die restlichen Graphen nicht, da dessen Knoten
	höchstens Grad 1 haben und somit nicht zu mind. zwei anderen adjazent sind.
\end{adjustwidth}
\textbf{(c)}
\begin{adjustwidth}{20pt}{20pt}
	Die Aussage des Satzes ist, dass im Graphen zwei verschiedene Knoten stets durch eine Kante verbunden sein sollen. (Das "$\exists(...)$" ist redundant, da innerhalb der Klammern
	$Exy$ nochmal in einer Konjunktion auftritt.) Dies trifft nur auf den Graphen  $\mathcal{G}_3$, da in allen anderen graphen Knoten existieren, welche nicht adjazent zueinander sind.
\end{adjustwidth}



\section*{Aufgabe 4 (Punkte:\qquad/\ppp)}
\textbf{(a)}
\begin{adjustwidth}{20pt}{20pt}
	$\varphi_a \defgr \exists x(a \cdot x = b)$
\end{adjustwidth}
\textbf{(b)}
\begin{adjustwidth}{20pt}{20pt}
	$\varphi_b \defgr \neg\exists x \neg\exists y(x \neq 1 \wedge y \neq 1 \wedge x \cdot y = a) \wedge a \neq 1$
\end{adjustwidth}
\textbf{(c)}
\begin{adjustwidth}{20pt}{20pt}
	$\varphi_c \defgr $
\end{adjustwidth}
\textbf{(d)}
\begin{adjustwidth}{20pt}{20pt}
	$\varphi_d \defgr $
\end{adjustwidth}
\textbf{(e)}
\begin{adjustwidth}{20pt}{20pt}
	$\varphi_e \defgr $
\end{adjustwidth}



\section*{Aufgabe 5 (Punkte:\qquad/\pppp)}
\textbf{(a)}
\begin{adjustwidth}{20pt}{20pt}
	
\end{adjustwidth}
\textbf{(b)}
\begin{adjustwidth}{20pt}{20pt}
	
\end{adjustwidth}



\end{document}
