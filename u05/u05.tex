\documentclass[11pt, a4paper]{article}

\usepackage{graphicx}
\usepackage{forest}
\usepackage{proof}
\usepackage[utf8]{inputenc}
\usepackage{fancyhdr}
\usepackage{changepage}
\usepackage[onehalfspacing]{setspace}
\usepackage{ragged2e}
\usepackage{ amssymb, amsmath, amsthm, dsfont }
\usepackage[width = 18cm, top = 2.5cm, bottom = 3cm]{geometry}
\usepackage{extarrows}
\usepackage{stmaryrd}
% ---------

\newcommand{\myTitleString} {}
\newcommand{\myAuthorString} {}
\newcommand{\mySubTitleString} {}
\newcommand{\myDateString} {}

\newcommand{\myTitle}[1] {\renewcommand {\myTitleString}{#1}}
\newcommand{\mySubTitle}[1] {\renewcommand {\mySubTitleString}{#1}}
\newcommand{\myAuthor}[1] {\renewcommand{\myAuthorString}		{#1}}
\newcommand{\myDate}[1] {\renewcommand{\myDateString}{#1}}

\newcommand{\makeMyTitle}
{
\pagestyle{fancy}
\fancyhead[L]
{
\begin{tabular}{l}
\myTitleString
\\ \mySubTitleString 
\\ \myDateString
\end{tabular}
} 			
\fancyhead[C]{}
\fancyhead[R]{\myAuthorString}
\fancyfoot[C]{\thepage}
}

\setlength{\headheight}{45pt}

\newcommand{\p}{7}
\newcommand{\pp}{5}
\newcommand{\ppp}{8}
\newcommand{\pppp}{6} 

\newcommand{\defgl}{\mathrel{=\!\!\mathop:}}
\newcommand{\defgr}{\mathrel{\mathop:\!\!=}}

\makeatletter
\renewcommand*\env@matrix[1][*\c@MaxMatrixCols c]{%
  \hskip -\arraycolsep
  \let\@ifnextchar\new@ifnextchar
  \array{#1}}
\makeatother
% ---------
%\setlength{\parindent}{0pt}
\begin{document}

\myTitle{\textsc{Mathematische Logik}}
\mySubTitle{Übung 5}
\myDate{21. Mai 2017}
\myAuthor
{
\begin{tabular}{l l}
346532, &Daniel Boschmann\\
348776, &Anton Beliankou	\\
356092, &Daniel Schleiz
\end{tabular}
}
\makeMyTitle

\begin{tabular}{|c|c|c|c|c|}\hline
   2 & 3 & 4 & 5 &$\sum$\\\hline
  	 \qquad/\p & \qquad/\pp & \qquad/\ppp & \qquad/\pppp &\qquad/26\\\hline
 \end{tabular}
\hspace*{20pt} {\huge Gruppe \textbf{G}}
\vspace*{30pt}


\section*{Aufgabe 2 (Punkte:\qquad/\p)}
\textbf{(a)}
\begin{adjustwidth}{20pt}{20pt}
	Es existieren die Redukte $(\mathbb{N},+,\cdot,<), (\mathbb{N},\cdot,<), (\mathbb{N},+,<), (\mathbb{N},+,\cdot), (\mathbb{N},<),
	(\mathbb{N},+), (\mathbb{N},\cdot)$ und $(\mathbb{N})$.
\end{adjustwidth}
\textbf{(b)}
\begin{adjustwidth}{20pt}{20pt}
	Für $n \subseteq \mathbb{N}$ und $n \neq \emptyset$ ist $\mathfrak{N}_n=(n,\leq)$ eine Substruktur von $\mathfrak{N}_1$, da sich jede Teilmenge der natürlichen Zahlen	
	auf die selbe Weise ordnen lässt. Dies sind alle möglichen Substrukturen.\\
	Sei $n \subseteq \mathbb{N}$, $n \neq \emptyset$ und sei $N_n=\{ 2^i \cdot m\ |\ m \in n, i \in \mathbb{N}\}$ sowie $N_k=\{ \mathbb{N}\backslash Q \ | \ Q = \{0,..,k\}, \ k \in \mathbb{N}\}$ die Teilmengen von $\mathbb{N}$ ohne Anfangselemente $0$ bis $k$. Ferner sei $N_{nk} \defgr N_2 \ \cup \ N_k$. Dann ist $\mathfrak{N}_{N_n}=(N_n, f)$ eine Substruktur
	von $\mathfrak{N}_2$, da man alle Elemente in $n$ um die Vielfachen mit allen Zweierpotenzen erweitern muss, da die Substruktur sonst nicht $\{ f \}$-abgeschlossen wäre. Ebenfalls ist $\mathfrak{N}_{N_{nk}}=(N_n, f)$
	eine Substruktur, da man ab einem gewissen $k$ auch alle Elemente der natürlichen Zahlen nehmen kann, die größer als $k$ sind. Für beliebige $k\in \mathbb{N}$ und $n \subseteq \mathbb{N}$, $n \neq \emptyset$
	sind dies alle möglichen Substrukturen.
\end{adjustwidth}
\textbf{(c)}
\begin{adjustwidth}{20pt}{20pt}
	$(\mathbb{Z}/3\mathbb{Z},+)$ besitzt die Substrukturen $\{0\}$ und $\{ 0,1,2 \}$, da $\{ 0,1 \}$ mit der Addition modulo 3 nicht abgeschlossen ist (1+1=2 liegt nicht in der Menge). \\
	$(\mathbb{Z}/4\mathbb{Z},+)$ besitzt die Substrukturen $\{0\}$ und $\{ 0,1,2,3 \}$ und $\{ 0,2 \}$. ($\{ 0,1\}$ nicht abgeschlossen wie vorher begründet, $\{ 0,1,2\}$ ebenfalls
	nicht, da $1+2=3$, $\{ 0,1,3\}$ nicht da $1+1=2$, $\{ 0,2\}$ und $\{ 0,3\}$ analog auch nicht.)
\end{adjustwidth}




\section*{Aufgabe 3 (Punkte:\qquad/\pp)}
\textbf{(a)}
\begin{adjustwidth}{20pt}{20pt}
	Die Aussage des Satzes ist, dass ein Knoten $x$ existiert, welcher nicht inzident zu einer Kante ist. Die Aussage trifft nur auf $\mathcal{G}_2$, der Knoten oben links ist isoliert.
\end{adjustwidth}
\textbf{(b)}
\begin{adjustwidth}{20pt}{20pt}
	Die Aussage des Satzes ist, dass ein Knoten $x$ existiert, welcher über eine Kante mit zwei anderen, verschiedenen, Knoten $y,z$ verbunden ist. Dies trifft auf die Graphen
	 $\mathcal{G}_1$ ($x$ ist Knoten oben links) und  $\mathcal{G}_4$ ($x$ ist z.B. der Knoten unten rechts) zu. Der Satz gilt für die restlichen Graphen nicht, da dessen Knoten
	höchstens Grad 1 haben und somit nicht zu mind. zwei anderen adjazent sind.
\end{adjustwidth}
\textbf{(c)}
\begin{adjustwidth}{20pt}{20pt}
	Die Aussage des Satzes ist, dass im Graphen zwei verschiedene Knoten stets durch eine Kante verbunden sein sollen. (Das "$\exists z(...)$" ist redundant, da innerhalb der Klammern
	$Exy$ nochmal in einer Konjunktion auftritt.) Dies trifft nur auf den Graphen  $\mathcal{G}_3$, da in allen anderen graphen Knoten existieren, welche nicht adjazent zueinander sind.
\end{adjustwidth}



\section*{Aufgabe 4 (Punkte:\qquad/\ppp)}
\textbf{(a)}
\begin{adjustwidth}{20pt}{20pt}
	$\varphi_{a}(a,b) \defgr \exists x(a \cdot x = b)$
\end{adjustwidth}
\textbf{(b)}
\begin{adjustwidth}{20pt}{20pt}
	$\varphi_{b}(a) \defgr \neg\exists x \neg\exists y(x \neq 1 \wedge y \neq 1 \wedge x \cdot y = a) \wedge a \neq 1$
\end{adjustwidth}
\textbf{(c)}
\begin{adjustwidth}{20pt}{20pt}
	$\varphi_{c}(a,b) \defgr \neg\exists n \neg\exists c \neg\exists d(n \cdot c = a \wedge n \cdot d = b)$
\end{adjustwidth}
\textbf{(d)}
\begin{adjustwidth}{20pt}{20pt}
	$\varphi_{d}(a) \defgr (a=1) \vee \forall x (\varphi_a(x,a) \wedge \varphi_b(x) \rightarrow 1+1=x)$
\end{adjustwidth}
\textbf{(e)}
\begin{adjustwidth}{20pt}{20pt}
	$\varphi_{e}(a,b) \defgr \forall x (\varphi_d(x) \rightarrow \forall y \forall z (((x+z \neq b) \rightarrow (x + y \neq a)) \wedge ((x+y\neq a) \rightarrow (x+z \neq b)))$
\end{adjustwidth}



\section*{Aufgabe 5 (Punkte:\qquad/\pppp)}
\textbf{(a)}
\begin{adjustwidth}{20pt}{20pt}
	Betrachte für den Induktionsanfang einen Term $t$, welcher nur aus einer Variable $x$ besteht. Dann gilt
	\[
	\llbracket t \rrbracket^{(\mathfrak{A},\beta)} =
	\llbracket x \rrbracket^{(\mathfrak{A},\beta)} = \beta(x) = \llbracket x \rrbracket^{(\mathfrak{B},\beta)} = \llbracket t \rrbracket^{(\mathfrak{B},\beta)}.
	\]
	Seien nun für den Induktionsschritt $t_1,...,t_n$ Terme für die die Aussage gilt und sei $f$ ein $n-$stelliges Funktionssymbol aus $\mathfrak{A}$ bzw. $\mathfrak{B}$. (Gleiche Signatur, da
	$\mathfrak{A}$ eine Substruktur.) Dann gilt
	\begin{align*}
	\llbracket ft_1...t_n \rrbracket^{(\mathfrak{A},\beta)} = f^{\mathfrak{A}}(\llbracket t_1 \rrbracket^{(\mathfrak{A},\beta)},...,\llbracket t_n\rrbracket^{(\mathfrak{A},\beta)})
	& \overset{IV}{=} f^{\mathfrak{A}}(\llbracket t_1 \rrbracket^{(\mathfrak{B},\beta)},...,\llbracket t_n\rrbracket^{(\mathfrak{B},\beta)}) \\
	& \overset{Def.}{=} f^{\mathfrak{B}}|_A(\llbracket t_1 \rrbracket^{(\mathfrak{B},\beta)},...,\llbracket t_n\rrbracket^{(\mathfrak{B},\beta)}) \\
	& = f^{\mathfrak{B}}(\llbracket t_1 \rrbracket^{(\mathfrak{B},\beta)},...,\llbracket t_n\rrbracket^{(\mathfrak{B},\beta)}) \\
	& = \llbracket ft_1...t_n \rrbracket^{(\mathfrak{B},\beta)}.
	\end{align*}
\end{adjustwidth}
\textbf{(b)}
\begin{adjustwidth}{20pt}{20pt}
	Sei $\mathfrak{A}$ eine Substruktur von $\mathfrak{B}$ und seien $a_1,...,a_k$ aus dem Universum von $\mathfrak{A}$. Zeige die Aussage induktiv über den Aufbau von FO Formeln.\\
	\textit{Induktionsanfang:} Seien Terme $t_i$ mit höchstens den Variablen $a_1,...,a_k$ aus dem Universum von $\mathfrak{A}$.
	\begin{itemize}
	\item Sei $\varphi=(t_1=t_2)$. Dann gilt $\mathfrak{A} \models \varphi \Leftrightarrow \llbracket t_1 \rrbracket^\mathfrak{A} = \llbracket t_2 \rrbracket^\mathfrak{A}
			\overset{(a)}{\Leftrightarrow} \llbracket t_1 \rrbracket^\mathfrak{B} = \llbracket t_2 \rrbracket^\mathfrak{B} \Leftrightarrow \mathfrak{B} \models \varphi$.
	\item Sei $\varphi=(Pt_1...t_n)$, wobei $P$ ein m-stelliges Relationssymbol aus $\mathfrak{A}$ (und somit aus $\mathfrak{B}$, da eine Substruktur die selbe Signatur hat.)\\
		Dann gilt $ \mathfrak{A} \models \varphi  \Leftrightarrow (\llbracket t_1 \rrbracket^\mathfrak{A},...,\llbracket t_n \rrbracket^\mathfrak{A}) \in P^{\mathfrak{A}}
		\overset{Def}{\Leftrightarrow} (\llbracket t_1 \rrbracket^\mathfrak{A},...,\llbracket t_n \rrbracket^\mathfrak{A}) \in P^{\mathfrak{B}} \cap A^m  \overset{(*)}{\Leftrightarrow } 
		(\llbracket t_1 \rrbracket^\mathfrak{A},...,\llbracket t_n \rrbracket^\mathfrak{A}) \in P^{\mathfrak{B}} \overset{(a)}{\Leftrightarrow }
		(\llbracket t_1 \rrbracket^\mathfrak{B},...,\llbracket t_n \rrbracket^\mathfrak{B}) \in P^{\mathfrak{B}} \Leftrightarrow \mathfrak{B} \models \varphi$
		[(*) Da $a_1,...,a_k$ aus dem Universum von $\mathfrak{A}$].
	\end{itemize}
	\textit{Induktionsschritt:} Die Behauptung gelte für Formeln $\psi, \varphi$, dessen Terme höchstens Variablen $a_1,...,a_k$ aus dem Universum von $\mathfrak{A}$ enthalten.
	\begin{itemize}
	\item $\mathfrak{A} \models \neg\varphi \Leftrightarrow \mathfrak{A} \not\models \varphi \overset{IV}{\Leftrightarrow} \mathfrak{B} \not\models \varphi \Leftrightarrow
		\mathfrak{B} \models \neg\varphi $
	\item $\mathfrak{A} \models (\varphi \vee \psi) \Leftrightarrow \mathfrak{A} \models \varphi \vee \mathfrak{A} \models \psi \overset{IV}{\Leftrightarrow}
		\mathfrak{B} \models \varphi \vee \mathfrak{B} \models \psi \Leftrightarrow \mathfrak{B} \models (\varphi \vee \psi)$
	\item $\mathfrak{A} \models (\varphi \wedge \psi) \Leftrightarrow \mathfrak{A} \models \varphi \wedge \mathfrak{A} \models \psi \overset{IV}{\Leftrightarrow}
		\mathfrak{B} \models \varphi \wedge \mathfrak{B} \models \psi \Leftrightarrow \mathfrak{B} \models (\varphi \wedge \psi)$
	\item $\mathfrak{A} \models (\varphi \rightarrow \psi) \Leftrightarrow \mathfrak{A} \not\models \varphi \vee \mathfrak{A} \models \psi \overset{IV}{\Leftrightarrow}
		\mathfrak{B} \not\models \varphi \vee \mathfrak{B} \models \psi \Leftrightarrow \mathfrak{B} \models (\varphi \rightarrow \psi)$
	\end{itemize}
Somit ist die Aussage für beliebige quantorenfreie FO-Formeln gezeigt. Es folgt insbesondere, da die Substruktur $\mathfrak{A}$ beliebig gewählt war, dass jede Substruktur von
$\mathfrak{B}$ die gleichen quantorenfreien Sätze erfüllt, da dies gerade die Rückrichtung der gezeigten Aussage ist.
\end{adjustwidth}



\end{document}
