\documentclass[11pt, a4paper]{article}

\usepackage{graphicx}
\usepackage{forest}
\usepackage{proof}
\usepackage[utf8]{inputenc}
\usepackage{fancyhdr}
\usepackage{changepage}
\usepackage[onehalfspacing]{setspace}
\usepackage{ragged2e}
\usepackage{ amssymb, amsmath, amsthm, dsfont }
\usepackage[width = 18cm, top = 2.5cm, bottom = 3cm]{geometry}
\usepackage{extarrows}
\usepackage{stmaryrd}
% ---------

\newcommand{\myTitleString} {}
\newcommand{\myAuthorString} {}
\newcommand{\mySubTitleString} {}
\newcommand{\myDateString} {}

\newcommand{\myTitle}[1] {\renewcommand {\myTitleString}{#1}}
\newcommand{\mySubTitle}[1] {\renewcommand {\mySubTitleString}{#1}}
\newcommand{\myAuthor}[1] {\renewcommand{\myAuthorString}		{#1}}
\newcommand{\myDate}[1] {\renewcommand{\myDateString}{#1}}

\newcommand{\makeMyTitle}
{
\pagestyle{fancy}
\fancyhead[L]
{
\begin{tabular}{l}
\myTitleString
\\ \mySubTitleString 
\\ \myDateString
\end{tabular}
} 			
\fancyhead[C]{}
\fancyhead[R]{\myAuthorString}
\fancyfoot[C]{\thepage}
}

\setlength{\headheight}{45pt}

\newcommand{\p}{14}
\newcommand{\pp}{3}
\newcommand{\ppp}{7}
\newcommand{\pppp}{6} 

\newcommand{\defgl}{\mathrel{=\!\!\mathop:}}
\newcommand{\defgr}{\mathrel{\mathop:\!\!=}}

\makeatletter
\renewcommand*\env@matrix[1][*\c@MaxMatrixCols c]{%
  \hskip -\arraycolsep
  \let\@ifnextchar\new@ifnextchar
  \array{#1}}
\makeatother
% ---------
%\setlength{\parindent}{0pt}
\begin{document}

\myTitle{\textsc{Mathematische Logik}}
\mySubTitle{Übung 6}
\myDate{29. Mai 2017}
\myAuthor
{
\begin{tabular}{l l}
346532, &Daniel Boschmann\\
348776, &Anton Beliankou	\\
356092, &Daniel Schleiz
\end{tabular}
}
\makeMyTitle

\begin{tabular}{|c|c|c|c|c|}\hline
   2 & 3 & 4 & 5 &$\sum$\\\hline
  	 \qquad/\p & \qquad/\pp & \qquad/\ppp & \qquad/\pppp &\qquad/30\\\hline
 \end{tabular}
\hspace*{20pt} {\huge Gruppe \textbf{G}}
\vspace*{30pt}


\section*{Aufgabe 2 (Punkte:\qquad/\p)}
\textbf{(a)}
\begin{adjustwidth}{20pt}{20pt}
	$\Phi_{\mathcal{K}_1} \defgr \{\forall x(Sx \rightarrow \exists y(Ty \wedge Rxy \wedge \forall z(Tz \wedge Rxz \rightarrow y=z))),
	\forall x\forall y\forall z(Rxz \wedge Ryz \rightarrow x=z), \forall y(Ty \rightarrow \exists x(Sx \wedge Rxy))\}$
\end{adjustwidth}
\textbf{(b)}
\begin{adjustwidth}{20pt}{20pt}
	Definiere $\varphi_n \defgr \exists x_1...\exists x_n\left(\bigwedge_{i=1,...,n}(\exists y(fx_i = y)) \wedge \bigwedge_{1\leq i \leq j \leq n}x_i \neq x_j\right)$.\\ Dann ist
	$\Phi_{\mathcal{K}_2} \defgr \{ \varphi_n\ |\ n \in \mathbb{N}\}$
\end{adjustwidth}
\textbf{(c)}
\begin{adjustwidth}{20pt}{20pt}
	Definiere $\psi_n \defgr \exists x_1,...,\exists x_n(\bigwedge_{i=1,...,n-1}(Rx_ix_{i+1}) \wedge Rx_nx_1)$. Dies ist erfüllt, wenn ein Kreis der Länge $n$ existiert.
	Dann ist das gesuchte Axiomensystem gegeben durch:\\
	$\Phi_{\mathcal{K}_3} \defgr \{ \neg\varphi_n\ |\ n \in \mathbb{N}\}$. (Nach Skript sind für gerichteten Graphen Schlingen erlaubt.)
\end{adjustwidth}
\textbf{(d)}
\begin{adjustwidth}{20pt}{20pt}
	$\Phi_{\mathcal{K}_4} \defgr \{ \forall x_0 \dotsc \forall x_n \left ( x_0 \wedge Sx_n = f(x_n) \rightarrow \left(\bigvee\limits_{i < n} \neg Rx_ix_{i+1} \right) \right ) \}$
\end{adjustwidth}
\textbf{(e)}
\begin{adjustwidth}{20pt}{20pt}
	$\Phi_{\mathcal{K}_5} \defgr \{\forall a(\neg Raa), \forall a\forall b \forall c(Rab \wedge Rbc \rightarrow Rac), \forall a \forall b(Rab \vee a=b \vee Rba), \forall x(Rxfx)\}$
\end{adjustwidth}
\textbf{(f)}
\begin{adjustwidth}{20pt}{20pt}
	$\Phi_{\mathcal{K}_6} \defgr \{ \forall x (Tx \rightarrow \exists y (Ty \wedge fy=x))\}$
\end{adjustwidth}
\textbf{(g)}
\begin{adjustwidth}{20pt}{20pt}
	$\Phi_{\mathcal{K}_7} \defgr \{ \forall x \forall y (Rxy \rightarrow \exists z (Sz \wedge fz=x \wedge Ty))\}$
\end{adjustwidth}




\section*{Aufgabe 3 (Punkte:\qquad/\pp)}
\textbf{(a)}
\begin{adjustwidth}{20pt}{20pt}
	
\end{adjustwidth}




\section*{Aufgabe 4 (Punkte:\qquad/\ppp)}
\textbf{(a)}
\begin{adjustwidth}{20pt}{20pt}

\end{adjustwidth}
\textbf{(b)}
\begin{adjustwidth}{20pt}{20pt}

\end{adjustwidth}
\textbf{(c)}
\begin{adjustwidth}{20pt}{20pt}

\end{adjustwidth}
\textbf{(d)}
\begin{adjustwidth}{20pt}{20pt}

\end{adjustwidth}
\textbf{(e)}
\begin{adjustwidth}{20pt}{20pt}

\end{adjustwidth}



\section*{Aufgabe 5 (Punkte:\qquad/\pppp)}
\textbf{(a)}
\begin{adjustwidth}{20pt}{20pt}
	
\end{adjustwidth}
\textbf{(b)}
\begin{adjustwidth}{20pt}{20pt}
	
\end{adjustwidth}



\end{document}
