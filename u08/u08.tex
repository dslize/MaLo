\documentclass[11pt, a4paper]{article}

\usepackage{graphicx}
\usepackage{forest}
\usepackage{proof}
\usepackage[utf8]{inputenc}
\usepackage{fancyhdr}
\usepackage{changepage}
\usepackage[onehalfspacing]{setspace}
\usepackage{ragged2e}
\usepackage{ amssymb, amsmath, amsthm, dsfont }
\usepackage[width = 18cm, top = 2.5cm, bottom = 3cm]{geometry}
\usepackage{extarrows}
\usepackage{stmaryrd}
% ---------

\newcommand{\myTitleString} {}
\newcommand{\myAuthorString} {}
\newcommand{\mySubTitleString} {}
\newcommand{\myDateString} {}

\newcommand{\myTitle}[1] {\renewcommand {\myTitleString}{#1}}
\newcommand{\mySubTitle}[1] {\renewcommand {\mySubTitleString}{#1}}
\newcommand{\myAuthor}[1] {\renewcommand{\myAuthorString}		{#1}}
\newcommand{\myDate}[1] {\renewcommand{\myDateString}{#1}}

\newcommand{\makeMyTitle}
{
\pagestyle{fancy}
\fancyhead[L]
{
\begin{tabular}{l}
\myTitleString
\\ \mySubTitleString 
\\ \myDateString
\end{tabular}
} 			
\fancyhead[C]{}
\fancyhead[R]{\myAuthorString}
\fancyfoot[C]{\thepage}
}

\setlength{\headheight}{45pt}

\newcommand{\p}{10}
\newcommand{\pp}{7}
\newcommand{\ppp}{15}
\newcommand{\pppp}{11*} 

\newcommand{\defgl}{\mathrel{=\!\!\mathop:}}
\newcommand{\defgr}{\mathrel{\mathop:\!\!=}}

\newcommand{\struc}[1]{\ensuremath{\mathfrak{#1}}}

\makeatletter
\renewcommand*\env@matrix[1][*\c@MaxMatrixCols c]{%
  \hskip -\arraycolsep
  \let\@ifnextchar\new@ifnextchar
  \array{#1}}
\makeatother
% ---------
%\setlength{\parindent}{0pt}
\begin{document}

\myTitle{\textsc{Mathematische Logik}}
\mySubTitle{Übung 8}
\myDate{17. Juni 2017}
\myAuthor
{
\begin{tabular}{l l}
346532, &Daniel Boschmann\\
348776, &Anton Beliankou	\\
356092, &Daniel Schleiz
\end{tabular}
}
\makeMyTitle

\begin{tabular}{|c|c|c|c|c|}\hline
   2 & 3 & 4 & 5 &$\sum$\\\hline
  	 \qquad/\p & \qquad/\pp & \qquad/\ppp & \qquad/\pppp &\qquad/32\\\hline
 \end{tabular}
\hspace*{20pt} {\huge Gruppe \textbf{G}}
\vspace*{30pt}


\section*{Aufgabe 2 (Punkte:\qquad/\p)}
\textbf{(a)}
\begin{adjustwidth}{20pt}{20pt}

\end{adjustwidth}
\textbf{(b)}
\begin{adjustwidth}{20pt}{20pt}
	Sei \struc{A} eine $\tau$-Struktur, sodass jedes Element elementar definierbar ist. Sei außerdem $\pi$ ein beliebiger Automorphismus von \struc{A} und $a \in A$ beliebig.
	Aus der elementaren Definierbarkeit von $a$ folgt, dass eine Formel $\varphi_a(x)$ existiert, sodass $\struc{A} \models \varphi_a(a)$ und $\struc{A} \not\models \varphi_a(b)$
	für alle $b \in A$ mit $b \neq a$. Da $\pi$ als Automorphismus auch insbesondere ein Isomorphismus ist, gilt mit dem Isomorphielemma, dass $\struc{A} \models \varphi_a(x)$ gdw.
	$\struc{A} \models \varphi_a(\pi(x))$. Nun muss aber $\pi(a)=a$ sein, da sonst das Isomorphielemma verletzt wäre. ($\struc{A} \models \varphi_a(a)$, aber
	$\struc{A} \not\models \varphi_a(\pi(a))$, falls $\pi(a) \neq a$.) \\
	Da $a$ beliebig gewählt war, gilt für alle $a \in A$, dass $\pi(a)=a$ und somit $\pi = 1_\struc{A}$, obwohl auch $\pi$ beliebig gewählt war. Es folgt also, dass nur $1_\struc{A}$
	ein Automorphismus von \struc{A} ist, also ist \struc{A} starr.
\end{adjustwidth}
\textbf{(c)}
\begin{adjustwidth}{20pt}{20pt}
	Eine unendliche Struktur mit dieser Eigenschaft ist zum Beispiel durch $(\mathbb{N},0,1,+)$ gegeben, da jedes Element elementar definierbar ist. (Sogar termdefinierbar durch
	$1+...+1$ für Zahlen größer als 1., 0 und 1 bereits in der Signatur.)
\end{adjustwidth}


\section*{Aufgabe 3 (Punkte:\qquad/\pp)}
\textbf{(a)}
\begin{adjustwidth}{20pt}{20pt}

\end{adjustwidth}
\textbf{(b)}
\begin{adjustwidth}{20pt}{20pt}

\end{adjustwidth}
\textbf{(c)}
\begin{adjustwidth}{20pt}{20pt}

\end{adjustwidth}

\section*{Aufgabe 4 (Punkte:\qquad/\ppp)}
\textbf{(a)}
\begin{adjustwidth}{20pt}{20pt}

\end{adjustwidth}
\textbf{(b)}
\begin{adjustwidth}{20pt}{20pt}

\end{adjustwidth}
\textbf{(c)}
\begin{adjustwidth}{20pt}{20pt}

\end{adjustwidth}
\textbf{(d)}
\begin{adjustwidth}{20pt}{20pt}

\end{adjustwidth}
\textbf{(e)}
\begin{adjustwidth}{20pt}{20pt}

\end{adjustwidth}
\textbf{(f)}
\begin{adjustwidth}{20pt}{20pt}

\end{adjustwidth}


\section*{Aufgabe 5 (Punkte:\qquad/\pppp)}
\textbf{(a)}
\begin{adjustwidth}{20pt}{20pt}

\end{adjustwidth}
\textbf{(b)}
\begin{adjustwidth}{20pt}{20pt}

\end{adjustwidth}
\textbf{(c)}
\begin{adjustwidth}{20pt}{20pt}

\end{adjustwidth}

\end{document}
