\documentclass[11pt, a4paper]{article}

\usepackage{graphicx}
\usepackage{forest}
\usepackage{proof}
\usepackage[utf8]{inputenc}
\usepackage{fancyhdr}
\usepackage{changepage}
\usepackage[onehalfspacing]{setspace}
\usepackage{ragged2e}
\usepackage{ amssymb, amsmath, amsthm, dsfont, marvosym }
\usepackage[width = 18cm, top = 2.5cm, bottom = 3cm]{geometry}
\usepackage{extarrows}
\usepackage{stmaryrd}
\usepackage{enumitem}
% ---------

\newcommand{\myTitleString} {}
\newcommand{\myAuthorString} {}
\newcommand{\mySubTitleString} {}
\newcommand{\myDateString} {}

\newcommand{\myTitle}[1] {\renewcommand {\myTitleString}{#1}}
\newcommand{\mySubTitle}[1] {\renewcommand {\mySubTitleString}{#1}}
\newcommand{\myAuthor}[1] {\renewcommand{\myAuthorString}		{#1}}
\newcommand{\myDate}[1] {\renewcommand{\myDateString}{#1}}

\newcommand{\makeMyTitle}
{
\pagestyle{fancy}
\fancyhead[L]
{
\begin{tabular}{l}
\myTitleString
\\ \mySubTitleString 
\\ \myDateString
\end{tabular}
} 			
\fancyhead[C]{}
\fancyhead[R]{\myAuthorString}
\fancyfoot[C]{\thepage}
}

\setlength{\headheight}{45pt}

\newcommand{\p}{10}
\newcommand{\pp}{7}
\newcommand{\ppp}{15}
\newcommand{\pppp}{11*} 

\newcommand{\defgl}{\mathrel{=\!\!\mathop:}}
\newcommand{\defgr}{\mathrel{\mathop:\!\!=}}

\newcommand{\struc}[1]{\ensuremath{\mathfrak{#1}}}

\makeatletter
\renewcommand*\env@matrix[1][*\c@MaxMatrixCols c]{%
  \hskip -\arraycolsep
  \let\@ifnextchar\new@ifnextchar
  \array{#1}}
\makeatother
% ---------
%\setlength{\parindent}{0pt}
\begin{document}

\myTitle{\textsc{Mathematische Logik}}
\mySubTitle{Übung 8}
\myDate{17. Juni 2017}
\myAuthor
{
\begin{tabular}{l l}
346532, &Daniel Boschmann\\
348776, &Anton Beliankou	\\
356092, &Daniel Schleiz
\end{tabular}
}
\makeMyTitle

\begin{tabular}{|c|c|c|c|c|}\hline
   2 & 3 & 4 & 5 &$\sum$\\\hline
  	 \qquad/\p & \qquad/\pp & \qquad/\ppp & \qquad/\pppp &\qquad/32\\\hline
 \end{tabular}
\hspace*{20pt} {\huge Gruppe \textbf{G}}
\vspace*{30pt}


\section*{Aufgabe 2 (Punkte:\qquad/\p)}
\textbf{(a)}
\begin{itemize}
\item \struc{Q} ist starr: 0 ist definierbar mit $\psi_0(a)=\forall x(a + x = x)$, 1 ist definierbar mit $\psi_1(a) \defgr \forall x(a \cdot x = x)$, -1 ist definierbar mit 
	$\psi_{-1}(a)=\exists x \exists y(\psi_0(x) \wedge \psi_1(y) \wedge a+1=0)$. Nun sind eine beliebige ganze Zahl $z \in \mathbb{Z}$ und $-z$ definierbar durch 
	$\psi_z(a)= \exists x \exists y(\psi_0(x) \wedge (\psi_1(y) \vee psi_{-1}(y)) \wedge a = x+y+...+y)$, wobei z-Mal in der Formel "+y" auftauchen soll. Nun ist ein beliebiger Bruch 
	$\frac{p}{q}$, $p,q \in \mathbb{Z}$ und sein negatives definierbar durch $\psi_{\frac{p}{q}}(a)=\exists x \exists y(\psi_p(x) \wedge \psi_q(y) \wedge y \cdot a = x)$. Da somit
	jedes Element aus dem Universum definierbar ist, gilt mit Aufgabenteil (b), dass \struc{Q} starr ist. 
\item (Schreibe der Übersichtlichkeit halber die Funktion +1 in Postfixnotation auf) \struc{N} ist starr: Zunächst ist die 0 definierbar durch $\psi_0(a)=\neg\exists x(x+1=a)$. Nun ist jede Zahl
	$n \in \mathbb{N}$ größer 0 definierbar durch $\psi_n(a)=\exists x(\psi_0(x) \wedge a=x(+1)^n)$, wobei $(+1)^n$ Kurzschreibweise für $+1+1...$ ist, für festes $n$ ist also $\psi_n$
	eine FO-Formel. Da jedes Element aus $\mathbb{N}$ elementar definierbar ist, ist mit Aufgabenteil (b) \struc{N} starr.
\item \struc{A} ist nicht starr, da zum Beispiel die Abbildung $\pi:z \mapsto z+6$ Automorphismus ist. Die Abbildung ist offensichtlich bijektiv, außerdem ist sie verträglich mit "<", da es
	eine Äquivalenzumformung ist, auf beiden Seiten einer Ungleichung mit "<" 6 zu addieren, somit gilt für beliebige $a,b \in \mathbb{Z}$, dass $a<b$ gdw. $\pi(a)<\pi(b)$. Zudem
	folgt aus $z \in 2\mathbb{Z}$, dass $\pi(z)=z+6 \in 2\mathbb{Z}$, da das Addieren einer gerade Zahl mit der geraden Zahl 6 erneut eine gerade Zahl ergibt. Außerdem 
	folgt aus $z \in 3\mathbb{Z}$, dass $\pi(z)=z+6 \in 3\mathbb{Z}$, weil 6 ein Vielfaches von 3 ist und das Aufaddieren von 6 auf ein Vielfaches von 3 ebenfalls Vielfaches von 3 ist.
	Da also ein Automorphismus ungleich der Identität existiert, ist \struc{A} nicht starr.
\end{itemize}
\textbf{(b)}
\begin{adjustwidth}{20pt}{20pt}
	Sei \struc{A} eine $\tau$-Struktur, sodass jedes Element elementar definierbar ist. Sei außerdem $\pi$ ein beliebiger Automorphismus von \struc{A} und $a \in A$ beliebig.
	Aus der elementaren Definierbarkeit von $a$ folgt, dass eine Formel $\varphi_a(x)$ existiert, sodass $\struc{A} \models \varphi_a(a)$ und $\struc{A} \not\models \varphi_a(b)$
	für alle $b \in A$ mit $b \neq a$. Da $\pi$ als Automorphismus auch insbesondere ein Isomorphismus ist, gilt mit dem Isomorphielemma, dass $\struc{A} \models \varphi_a(x)$ gdw.
	$\struc{A} \models \varphi_a(\pi(x))$. Nun muss aber $\pi(a)=a$ sein, da sonst das Isomorphielemma verletzt wäre. ($\struc{A} \models \varphi_a(a)$, aber
	$\struc{A} \not\models \varphi_a(\pi(a))$, falls $\pi(a) \neq a$.) \\
	Da $a$ beliebig gewählt war, gilt für alle $a \in A$, dass $\pi(a)=a$ und somit $\pi = 1_\struc{A}$, obwohl auch $\pi$ beliebig gewählt war. Es folgt also, dass nur $1_\struc{A}$
	ein Automorphismus von \struc{A} ist, also ist \struc{A} starr.
\end{adjustwidth}
\textbf{(c)}
\begin{adjustwidth}{20pt}{20pt}
	Eine unendliche Struktur mit dieser Eigenschaft ist zum Beispiel durch $(\mathbb{N},0,1,+)$ gegeben, da jedes Element elementar definierbar ist. (Sogar termdefinierbar durch
	$1+...+1$ für Zahlen größer als 1., 0 und 1 bereits in der Signatur.)
\end{adjustwidth}


\section*{Aufgabe 3 (Punkte:\qquad/\pp)}
\textbf{(a)}
\begin{adjustwidth}{20pt}{20pt}
	Angenommen, $\pi\upharpoonright\mathbb{P}$ ist keine Primzahlpermutation. Da $\pi$ Automorphismus und somit insbesondere injektiv ist, existieren also
	$p \in \mathbb{P}$ und $n \in \mathbb{N}\backslash \mathbb{P}$, sodass $\pi(p)=n$. Betrachte nun die Formel $\varphi_{\text{prim}}(a)$ aus Aufgabenteil 4(b), welche
	wahr ist gdw. a prim ist. Nach Isomorphielemma müsste aus $(\mathbb{N},\cdot) \models \varphi_{\text{prim}}(p)$ ebenfalls
	$(\mathbb{N},\cdot) \models \varphi_{\text{prim}}(\pi(p))$ folgen, dies steht aber im Widerspruch dazu, dass $\pi(p)=n$ nicht prim ist. Somit war die Annahme falsch, dass
	$\pi\upharpoonright\mathbb{P}$ keine Primzahlpermutation ist.
\end{adjustwidth}
\textbf{(b)}
\begin{adjustwidth}{20pt}{20pt}
	Sei $f:\mathbb{P}\to \mathbb{P}$ eine Primzahlpermutation. Zeige zunächst, dass der Automorphismus $\pi_f:\mathbb{N} \to \mathbb{N}$ existiert. Definiere dazu $\pi_f$ wie
	folgt:
	\[
	\pi_f(x) \defgr 
	\begin{cases}
	f(x) & \text{falls $x\in \mathbb{P}$}\\
	x & \text{falls $x \in \{0,1\}$}\\
	f(p_1)\cdot...\cdot f(p_n) & \text{falls $x\notin \mathbb{P}$ und mit der Primfaktorzerlegung $p_1,...,p_n$ von $x$}
	\end{cases}
	\]
	Die Abbildung ist wohldefiniert, da jede natürliche Zahl eine eindeutige Primfaktorzerlegung besitzt. 
	Außerdem ist aufgrund des ersten Falles in der Definition $\pi_f \upharpoonright \mathbb{P}=f$. Die Abbildung ist bijektiv: Injektiv, da jede natürliche Zahl eine eindeutige und
	einzigartige Primfaktorzerlegung besitzt (bis auf die Reihenefolge der Faktoren) und die Primzahlpermutation bijektiv und dadurch insbesondere injektiv ist, weshalb auf jede Zahl
	höchstens einmal abgebildet wird. Surjektiv, da man für eine beliebige Zahl dessen Primfaktorzerlegung berechnen kann und durch Anwendung der bijektiven Primzahlpermutation $f$ auf die
	Faktoren und anschließender Produktbildung wieder auf das (einelementige) Urbild schließen kann.\\
	Zudem ist $\pi_f$ verträglich mit der Multiplikation, denn: Seien $a,b \in \mathbb{N}$ beliebig und $p_1,...,p_k,q_1,...,q_l \in \mathbb{P}$ die Primfaktoren von $a$ bzw. $b$, d.h.
	${\displaystyle a=\prod_{1 \leq i \leq k}p_i}$ und ${\displaystyle b=\prod_{1 \leq j \leq l}q_j}$. Stelle zunächst fest, dass daraus die Primfaktorzerlegung für $a \cdot b$ folgt, da
	$a \cdot b = {\displaystyle \prod_{1 \leq i \leq k}p_i} \cdot {\displaystyle \prod_{1 \leq j \leq l}q_j}$. Nun gilt:
	\[
	\pi(a) \cdot \pi(b) = \prod_{1 \leq i \leq k}f(p_i) \cdot \prod_{1 \leq j \leq l}f(q_j) = \pi(a \cdot b)
	\]
	Für die Sonderfälle 0 und 1, welche keine Primfaktorzerlegung besitzen, ist die Abbildung ebenfalls verträglich mit der Multiplikation, denn für $c \in \mathbb{N}$ beliebig ist
	$\pi(c) \cdot \pi(1) = \pi(c) \cdot 1 = \pi(c \cdot 1)$ und $\pi(c) \cdot \pi(0) = \pi(c) \cdot 0 = 0 = \pi(0) = \pi(c \cdot 0)$. Es folgt nun, dass $\pi_f$ ein Automorphismus ist.\\
	Angenommen, es existiert ein anderer Automorphismus $\pi_f'$, sodass $\pi_f' \upharpoonright \mathbb{P}=f$. Dann werden Primzahlen gemäß der Permutation abgebildet, zudem müssen 0 und
	1 festgehalten werden, da diese elementar definierbar sind (mittels $\psi_0(a)=\forall x(a\cdot x=a)$ und $\psi_1(a)=\forall x(a \cdot x=x)$), und sonst ein Widerspruch zum Isomorphielemma
	entstehen würde. Es existiert also ein $z \in \mathbb{N}$, sodass $\pi(z) \neq f(p_1)\cdot...\cdot f(p_n)$, wobei $p_1,...,p_n$ die Primfaktorzerlegung von $z$ sei. Dann gilt 
	$\pi(z)=\pi(p_1 \cdot ... \cdot p_n) \neq \pi(p_1) \cdot ... \cdot \pi(p_n)$, Widerspruch dazu, dass $\pi$ ein Isomorphismus ist. Somit gilt $\pi_f' = \pi_f$, also eindeutig. 
\end{adjustwidth}
\textbf{(c)}
\begin{adjustwidth}{20pt}{20pt}
	Die durch die Primzahlpermutationen induzierten Automorphismen von $(\mathbb{N},\cdot)$ charakterisieren auch die Automorphismen von $(\mathbb{N},\cdot,|)$, denn: Stelle zunächst einmal
	fest, dass durch Erweiterung der Signatur höchstens Automorphismen wegfallen könnten durch Unverträglichkeit mit der neuen Relation, und keine hinzukommen können, aufgrund der Verträglichkeit
	mit $\cdot$. Betrachte einen Automorhpismus $\pi$ wie in (a) und (b) beschrieben und seien $m,n \in \mathbb{N}$ gegeben. Da 0 und 1 festgehalten werden, kann in solchen Fällen kein
	Widerspruch entstehen. Zudem ist, falls $m|n$, $n$ nie prim, außer $m=1$ oder $m=n$. (In den Fällen gilt $\pi(m)|\pi(n)$.) Sei nun $m|n$ und $n$ nicht prim, d.h. es existiert eine
	Primfaktorzerlegung $p_1,...,p_k$ von $n$. Da $n$ von $m$ geteilt wird, ist entweder $m$ gleich einer der Primfaktoren von $n$ oder $m$ ist ebenfalls nicht prim, jedoch bilden die
	Primfaktoren von $m$ eine Teilmenge von denen von $n$, da sonst die Teilbarkeitsrelation nicht erfüllt wäre. Nun folgt, dass $\pi(m)\pi(n)$, da ein Permutieren eines Primfaktors, welcher 		sowohl in $m$ als auch in $n$ auftritt, nichts an der Teilbarkeit ändert. Gleiches gilt, falls nur die Primfaktoren permutiert werden, welche in $n$ aber nicht in $m$ auftauchen.
	Es folgt also nun die Aussage.
\end{adjustwidth}

\section*{Aufgabe 4 (Punkte:\qquad/\ppp)}
\textbf{(a)}
\begin{adjustwidth}{20pt}{20pt}
	Elementar definierbar, durch die Formel: $\varphi(x) \defgr \forall y(y \cdot x = y)$.
\end{adjustwidth}
\textbf{(b)}
\begin{adjustwidth}{20pt}{20pt}
TODO
\end{adjustwidth}
\textbf{(c)}
\begin{adjustwidth}{20pt}{20pt}
	Nicht elementar definierbar: Wähle als Automorphismus den durch die Primpermutation, welche 5 auf 3 und 3 auf 5 abbildet und die restlichen Primzahlen festhält, induzierten Automorphismus.
	Angenommen, es existiert eine Formel $\psi(x,y)$, welche die gegebene Relation definiert. Dann gilt $(\mathbb{N},\cdot) \models (2,5\cdot 2)$ und mit dem Isomorphielemma müsste folgen, dass
	auch $(\mathbb{N},\cdot) \models (\pi(2),\pi(5\cdot 2))=(2,3 \cdot 2)$, es ist aber $5\cdot 2 \neq 3 \cdot 2$, also Widerspruch. Damit ist die Relation nicht definierbar.
\end{adjustwidth}
\textbf{(d)}
\begin{adjustwidth}{20pt}{20pt}
	Die Relation ist elementar definierbar. Definiere zunächst ein paar Hilfsformeln:\\
	$\varphi_{\text{teilt}}(a,b) \defgr \exists x(a \cdot x = b)$. ("a teilt b") und\\
	$\varphi_{\text{prim}}(a) \defgr \forall x(\varphi_{\text{teilt}}(x,a) \rightarrow x=1 \vee x=a) \wedge a\neq 1$ ("a ist prim").\\
	Nun lässt sich die gewünschte Relation durch folgende Formel definieren:\\
	$\varphi_{pp}(a) \defgr \exists x(\varphi_{\text{teilt}}(x,a) \wedge \varphi_{\text{prim}}(x) \wedge \forall y(\varphi_{\text{teilt}}(y,a) \rightarrow y=x))$
\end{adjustwidth}
\textbf{(e)}
\begin{adjustwidth}{20pt}{20pt}
	Die Relation ist nicht elementar definierbar. Definiere dazu zunächst auf den komplexen Zahlen den Automorphismus $\pi:\mathbb{C} \to \mathbb{C}, a+bi \mapsto a-bi$. Die Abbildung
	ist sowohl surjektiv (Für eine beliebige komplexe Zahl wird dessen komplex konjugierte Zahl darauf abgebildet.) und ebenfalls injektiv (Der Realteil bleibt identisch, durch negieren des
	Imaginärteils können nicht zwei verschiedene komplexe Zahlen auf dieselbe Zahl abbilden.)\\
	Angenommen, es gäbe $\varphi(x) \in \text{FO}(\{ +\})$, welche die Menge elementar definiert. Dann gilt für $z \defgr 2+2i$, dass $(\mathbb{C},+) \models \varphi(z)$, da
	$\operatorname{Re}(z)=2=\operatorname{Im}(z)$. Dann gilt mit dem Isomorphielemma, dass auch $(\mathbb{C},+) \models \varphi(\pi(z))=\varphi(2-2i)$. \Lightning\ Widerspruch, da
	$\operatorname{Re}(\pi(z))=2\neq -2=\operatorname{Im}(\pi(z))$. Somit ist die Relation nicht elementar definierbar.
\end{adjustwidth}
\textbf{(f)}
\begin{adjustwidth}{20pt}{20pt}
	Die Relation ist elementar definierbar. Definiere zunächst die Hilfsformel $\psi_\epsilon(a) \defgr \forall x(a \preceq x)$. Diese Formel definiert gerade die Menge mit dem leeren Wort, also
	$\{\epsilon\}$, da nur das leere Wort Präfix jedes Wortes ist. Definiere nun damit die gewünschte Relation:\\
	$\psi_{01}(a) \defgr \neg\psi_\epsilon(a) \wedge \forall y(y \preceq a \rightarrow \psi_\epsilon(y) \vee y=a)$.
\end{adjustwidth}


\section*{Aufgabe 5 (Punkte:\qquad/\pppp)}
\textbf{(a)} Seien $\varphi(x_1,...,x_k) \in \text{FO}(\tau)$ und $a_1,...,a_k \in A$ beliebig.
\begin{enumerate}[label=(\roman*)]
\item Sei $\struc{A} \subseteq \struc{B} \preceq \struc{C}$ und $\struc{A} \preceq \struc{C}$. Dann gilt:
	$$\bigg( \struc{A} \models \varphi(a_1,...,a_k) \bigg) \overset{\struc{A} \preceq \struc{C}}{\Leftrightarrow} \bigg( \struc{C} \models \varphi(a_1,...,a_k) \bigg)
	\overset{\struc{B} \preceq \struc{C}}{\Leftrightarrow} \bigg( \struc{B} \models \varphi(a_1,...,a_k) \bigg)$$
Es folgt also, dass $\struc{A} \preceq \struc{B}$.
\item Die Aussage gilt nicht. Betrachte die Strukturen $(\mathbb{N},\cdot)$ und $(2\mathbb{N},\cdot)$, wobei $2\mathbb{N} \defgr \{2n\ |\ n \in \mathbb{N}\}$ die geraden Zahlen sind.
Definiere den Isomorphismus $\pi:2\mathbb{N} \to \mathbb{N}, n \mapsto \frac{n}{2}$. (Offensichtlich bijektiv.) Damit gilt $\mathbb{N} \cong 2\mathbb{N}$.
Zunächst ist die 0 definierbar durch $\psi_0(a)=\forall x(a + x = x)$. Betrachte die Formel $\psi(a)=\neg\psi_0(a) \wedge \neg \exists x \exists y(\neg\psi_0(x) \wedge \neg\psi_0(y) \wedge
x+y=a)$ ("$a$ ist ungleich 0 und es existieren keine zwei Zahlen ungleich 0, sodass dessen Summe $a$ ergibt.") Nun gilt $(2\mathbb{N},\cdot) \models \psi(2)$, da 2 in $2\mathbb{N}$
nicht als Summe von zwei Zahlen ungleich 0 dargestellt werden kann, jedoch ist $(2\mathbb{N},\cdot) \not\models \psi(2)$, da in $\mathbb{N}$ gilt, dass $2=1+1$. Widerspruch zum
Isomophielemma, also ist die Aussage falsch.
\end{enumerate}
\textbf{(b)}
\begin{adjustwidth}{20pt}{20pt}

\end{adjustwidth}
\textbf{(c)}
\begin{adjustwidth}{20pt}{20pt}

\end{adjustwidth}

\end{document}
