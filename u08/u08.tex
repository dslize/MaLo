\documentclass[11pt, a4paper]{article}

\usepackage{graphicx}
\usepackage{forest}
\usepackage{proof}
\usepackage[utf8]{inputenc}
\usepackage{fancyhdr}
\usepackage{changepage}
\usepackage[onehalfspacing]{setspace}
\usepackage{ragged2e}
\usepackage{ amssymb, amsmath, amsthm, dsfont, marvosym }
\usepackage[width = 18cm, top = 2.5cm, bottom = 3cm]{geometry}
\usepackage{extarrows}
\usepackage{stmaryrd}
% ---------

\newcommand{\myTitleString} {}
\newcommand{\myAuthorString} {}
\newcommand{\mySubTitleString} {}
\newcommand{\myDateString} {}

\newcommand{\myTitle}[1] {\renewcommand {\myTitleString}{#1}}
\newcommand{\mySubTitle}[1] {\renewcommand {\mySubTitleString}{#1}}
\newcommand{\myAuthor}[1] {\renewcommand{\myAuthorString}		{#1}}
\newcommand{\myDate}[1] {\renewcommand{\myDateString}{#1}}

\newcommand{\makeMyTitle}
{
\pagestyle{fancy}
\fancyhead[L]
{
\begin{tabular}{l}
\myTitleString
\\ \mySubTitleString 
\\ \myDateString
\end{tabular}
} 			
\fancyhead[C]{}
\fancyhead[R]{\myAuthorString}
\fancyfoot[C]{\thepage}
}

\setlength{\headheight}{45pt}

\newcommand{\p}{10}
\newcommand{\pp}{7}
\newcommand{\ppp}{15}
\newcommand{\pppp}{11*} 

\newcommand{\defgl}{\mathrel{=\!\!\mathop:}}
\newcommand{\defgr}{\mathrel{\mathop:\!\!=}}

\newcommand{\struc}[1]{\ensuremath{\mathfrak{#1}}}

\makeatletter
\renewcommand*\env@matrix[1][*\c@MaxMatrixCols c]{%
  \hskip -\arraycolsep
  \let\@ifnextchar\new@ifnextchar
  \array{#1}}
\makeatother
% ---------
%\setlength{\parindent}{0pt}
\begin{document}

\myTitle{\textsc{Mathematische Logik}}
\mySubTitle{Übung 8}
\myDate{17. Juni 2017}
\myAuthor
{
\begin{tabular}{l l}
346532, &Daniel Boschmann\\
348776, &Anton Beliankou	\\
356092, &Daniel Schleiz
\end{tabular}
}
\makeMyTitle

\begin{tabular}{|c|c|c|c|c|}\hline
   2 & 3 & 4 & 5 &$\sum$\\\hline
  	 \qquad/\p & \qquad/\pp & \qquad/\ppp & \qquad/\pppp &\qquad/32\\\hline
 \end{tabular}
\hspace*{20pt} {\huge Gruppe \textbf{G}}
\vspace*{30pt}


\section*{Aufgabe 2 (Punkte:\qquad/\p)}
\textbf{(a)}
\begin{adjustwidth}{20pt}{20pt}

\end{adjustwidth}
\textbf{(b)}
\begin{adjustwidth}{20pt}{20pt}
	Sei \struc{A} eine $\tau$-Struktur, sodass jedes Element elementar definierbar ist. Sei außerdem $\pi$ ein beliebiger Automorphismus von \struc{A} und $a \in A$ beliebig.
	Aus der elementaren Definierbarkeit von $a$ folgt, dass eine Formel $\varphi_a(x)$ existiert, sodass $\struc{A} \models \varphi_a(a)$ und $\struc{A} \not\models \varphi_a(b)$
	für alle $b \in A$ mit $b \neq a$. Da $\pi$ als Automorphismus auch insbesondere ein Isomorphismus ist, gilt mit dem Isomorphielemma, dass $\struc{A} \models \varphi_a(x)$ gdw.
	$\struc{A} \models \varphi_a(\pi(x))$. Nun muss aber $\pi(a)=a$ sein, da sonst das Isomorphielemma verletzt wäre. ($\struc{A} \models \varphi_a(a)$, aber
	$\struc{A} \not\models \varphi_a(\pi(a))$, falls $\pi(a) \neq a$.) \\
	Da $a$ beliebig gewählt war, gilt für alle $a \in A$, dass $\pi(a)=a$ und somit $\pi = 1_\struc{A}$, obwohl auch $\pi$ beliebig gewählt war. Es folgt also, dass nur $1_\struc{A}$
	ein Automorphismus von \struc{A} ist, also ist \struc{A} starr.
\end{adjustwidth}
\textbf{(c)}
\begin{adjustwidth}{20pt}{20pt}
	Eine unendliche Struktur mit dieser Eigenschaft ist zum Beispiel durch $(\mathbb{N},0,1,+)$ gegeben, da jedes Element elementar definierbar ist. (Sogar termdefinierbar durch
	$1+...+1$ für Zahlen größer als 1., 0 und 1 bereits in der Signatur.)
\end{adjustwidth}


\section*{Aufgabe 3 (Punkte:\qquad/\pp)}
\textbf{(a)}
\begin{adjustwidth}{20pt}{20pt}
	Angenommen, $\pi\upharpoonright\mathbb{P}$ ist keine Primzahlpermutation. Da $\pi$ Automorphismus und somit insbesondere injektiv ist, existieren also
	$p \in \mathbb{P}$ und $n \in \mathbb{N}\backslash \mathbb{P}$, sodass $\pi(p)=n$. Betrachte nun die Formel $\varphi_{\text{prim}}(a)$ aus Aufgabenteil 4(b), welche
	wahr ist gdw. a prim ist. Nach Isomorphielemma müsste aus $(\mathbb{N},\cdot) \models \varphi_{\text{prim}}(p)$ ebenfalls
	$(\mathbb{N},\cdot) \models \varphi_{\text{prim}}(\pi(p))$ folgen, dies steht aber im Widerspruch dazu, dass $\pi(p)=n$ nicht prim ist. Somit war die Annahme falsch, dass
	$\pi\upharpoonright\mathbb{P}$ keine Primzahlpermutation ist.
\end{adjustwidth}
\textbf{(b)}
\begin{adjustwidth}{20pt}{20pt}
	Sei $f:\mathbb{P}\to \mathbb{P}$ eine Primzahlpermutation. Zeige zunächst, dass der Automorphismus $\pi_f:\mathbb{N} \to \mathbb{N}$ existiert. Definiere dazu $\pi_f$ wie
	folgt:
	\[
	\pi_f(x) \defgr 
	\begin{cases}
	f(x) & \text{falls $x\in \mathbb{P}$}\\
	x & \text{falls $x \in \{0,1\}$}\\
	f(p_1)\cdot...\cdot f(p_n) & \text{falls $x\notin \mathbb{P}$ und mit der Primfaktorzerlegung $p_1,...,p_n$ von $x$}
	\end{cases}
	\]
	Die Abbildung ist wohldefiniert, da jede natürliche Zahl eine eindeutige Primfaktorzerlegung besitzt. Die Abbildung ist bijektiv: Injektiv, da jede natü
\end{adjustwidth}
\textbf{(c)}
\begin{adjustwidth}{20pt}{20pt}

\end{adjustwidth}

\section*{Aufgabe 4 (Punkte:\qquad/\ppp)}
\textbf{(a)}
\begin{adjustwidth}{20pt}{20pt}
	Elementar definierbar, durch die Formel: $\varphi(x) \defgr \forall y(y \cdot x = y)$.
\end{adjustwidth}
\textbf{(b)}
\begin{adjustwidth}{20pt}{20pt}

\end{adjustwidth}
\textbf{(c)}
\begin{adjustwidth}{20pt}{20pt}
	Elementar definierbar, durch die Formel: $\varphi_c(x,y) \defgr n = m \cdot m \cdot m \cdot m \cdot m$.
\end{adjustwidth}
\textbf{(d)}
\begin{adjustwidth}{20pt}{20pt}
	Die Relation ist elementar definierbar. Definiere zunächst ein paar Hilfsformeln:\\
	$\varphi_{\text{teilt}}(a,b) \defgr \exists x(a \cdot x = b)$. ("a teilt b") und\\
	$\varphi_{\text{prim}}(a) \defgr \forall x(\varphi_{\text{teilt}}(x,a) \rightarrow x=1 \vee x=a) \wedge a\neq 1$ ("a ist prim").\\
	Nun lässt sich die gewünschte Relation durch folgende Formel definieren:\\
	$\varphi_{pp}(a) \defgr \exists x(\varphi_{\text{teilt}}(x,a) \wedge \varphi_{\text{prim}}(x) \wedge \forall y(\varphi_{\text{teilt}}(y,a) \rightarrow y=x))$
\end{adjustwidth}
\textbf{(e)}
\begin{adjustwidth}{20pt}{20pt}
	Die Relation ist nicht elementar definierbar. Definiere dazu zunächst auf den komplexen Zahlen den Automorphismus $\pi:\mathbb{C} \to \mathbb{C}, a+bi \mapsto a-bi$. Die Abbildung
	ist sowohl surjektiv (Für eine beliebige komplexe Zahl wird dessen komplex konjugierte Zahl darauf abgebildet.) und ebenfalls injektiv (Der Realteil bleibt identisch, durch negieren des
	Imaginärteils können nicht zwei verschiedene komplexe Zahlen auf dieselbe Zahl abbilden.)\\
	Angenommen, es gäbe $\varphi(x) \in \text{FO}(\{ +\})$, welche die Menge elementar definiert. Dann gilt für $z \defgr 2+2i$, dass $(\mathbb{C},+) \models \varphi(z)$, da
	$\operatorname{Re}(z)=2=\operatorname{Im}(z)$. Dann gilt mit dem Isomorphielemma, dass auch $(\mathbb{C},+) \models \varphi(\pi(z))=\varphi(2-2i)$. \Lightning\ Widerspruch, da
	$\operatorname{Re}(\pi(z))=2\neq -2=\operatorname{Im}(\pi(z))$. Somit ist die Relation nicht elementar definierbar.
\end{adjustwidth}
\textbf{(f)}
\begin{adjustwidth}{20pt}{20pt}
	Die Relation ist elementar definierbar. Definiere zunächst die Hilfsformel $\psi_\epsilon(a) \defgr \forall x(a \preceq x)$. Diese Formel definiert gerade die Menge mit dem leeren Wort, also
	$\{\epsilon\}$, da nur das leere Wort Präfix jedes Wortes ist. Definiere nun damit die gewünschte Relation:\\
	$\psi_{01}(a) \defgr \neg\psi_\epsilon(a) \wedge \forall y(y \preceq a \rightarrow \psi_\epsilon(y) \vee y=a)$.
\end{adjustwidth}


\section*{Aufgabe 5 (Punkte:\qquad/\pppp)}
\textbf{(a)}
\begin{adjustwidth}{20pt}{20pt}
% einfach! TODO
\end{adjustwidth}
\textbf{(b)}
\begin{adjustwidth}{20pt}{20pt}

\end{adjustwidth}
\textbf{(c)}
\begin{adjustwidth}{20pt}{20pt}

\end{adjustwidth}

\end{document}
