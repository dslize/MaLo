\documentclass[11pt, a4paper]{article}

\usepackage{graphicx}
\usepackage{forest}
\usepackage{proof}
\usepackage[utf8]{inputenc}
\usepackage{fancyhdr}
\usepackage{changepage}
\usepackage[onehalfspacing]{setspace}
\usepackage{ragged2e}
\usepackage{ amssymb, amsmath, amsthm, dsfont, marvosym }
\usepackage[width = 18cm, top = 2.5cm, bottom = 3cm]{geometry}
\usepackage{extarrows}
\usepackage{stmaryrd}
\usepackage{enumitem}
% ---------

\newcommand{\myTitleString} {}
\newcommand{\myAuthorString} {}
\newcommand{\mySubTitleString} {}
\newcommand{\myDateString} {}

\newcommand{\myTitle}[1] {\renewcommand {\myTitleString}{#1}}
\newcommand{\mySubTitle}[1] {\renewcommand {\mySubTitleString}{#1}}
\newcommand{\myAuthor}[1] {\renewcommand{\myAuthorString}		{#1}}
\newcommand{\myDate}[1] {\renewcommand{\myDateString}{#1}}

\newcommand{\makeMyTitle}
{
\pagestyle{fancy}
\fancyhead[L]
{
\begin{tabular}{l}
\myTitleString
\\ \mySubTitleString 
\\ \myDateString
\end{tabular}
} 			
\fancyhead[C]{}
\fancyhead[R]{\myAuthorString}
\fancyfoot[C]{\thepage}
}

\setlength{\headheight}{45pt}

\newcommand{\p}{7}
\newcommand{\pp}{11}
\newcommand{\ppp}{8}
\newcommand{\pppp}{} 

\newcommand{\defgl}{\mathrel{=\!\!\mathop:}}
\newcommand{\defgr}{\mathrel{\mathop:\!\!=}}

\newcommand{\struc}[1]{\ensuremath{\mathfrak{#1}}}

\makeatletter
\renewcommand*\env@matrix[1][*\c@MaxMatrixCols c]{%
  \hskip -\arraycolsep
  \let\@ifnextchar\new@ifnextchar
  \array{#1}}
\makeatother
% ---------
%\setlength{\parindent}{0pt}
\begin{document}

\myTitle{\textsc{Mathematische Logik}}
\mySubTitle{Übung 9}
\myDate{26. Juni 2017}
\myAuthor
{
\begin{tabular}{l l}
346532, &Daniel Boschmann\\
348776, &Anton Beliankou	\\
356092, &Daniel Schleiz
\end{tabular}
}
\makeMyTitle

\begin{tabular}{|c|c|c|c|}\hline
   2 & 3 & 4 &$\sum$\\\hline
  	 \qquad/\p & \qquad/\pp & \qquad/\ppp &\qquad/26\\\hline
 \end{tabular}
\hspace*{20pt} {\huge Gruppe \textbf{G}}
\vspace*{30pt}


\section*{Aufgabe 2 (Punkte:\qquad/\p)}
\textbf{(a)}
\begin{adjustwidth}{20pt}{20pt}
	\struc{A} und \struc{B} sind elementar äquivalent, da die Duplikatorin in $G(\struc{A},\struc{B})$ stets eine Gewinnstrategie hat. Da in beiden Strukturen unendlich viele Elemente $x$
	existieren, die in der Relation $P$ liegen. (Und ebenso welche, die nicht in der Relation liegen.) Als Gewinnstrategie könnte die Duplikatorin zum Beispiel für eine nicht-negative Zahl aus
	der ersten Struktur (vom Herausforderer gewählt) aus der zweiten Struktur Zahlen in der Reihenfolge -1,-2,-3,... geben (falls eine solche Zahl in einem anderen Schritt vom erausforderer bereits gewählt wurde, wird
	einfach die nächste Zahl der Folge gewählt), bei einer negativen Zahl eben positive Zahlen in der zweiten Struktur in der Folge 1,2,3,... (Analoge Strategie, falls der Herausforderer aus
	der zweiten Struktur wählt, also genau andersrum.) Somit sind die Strukturen elementar äquivalent.
	
\end{adjustwidth}
\textbf{(b)}
\begin{adjustwidth}{20pt}{20pt}
	Es gilt $\struc{A} \equiv_1 \struc{B}$, aber $\struc{A} \not\equiv_2 \struc{B}$. Trennende Formel: $\psi = \exists x \exists y(x \neq y \wedge Px \wedge Qx \wedge Py \wedge Qy)$, es ist
	qr($\psi$)=2. Die Formel sagt also aus, dass es mindestens zwei Elemente gibt, welche in beiden Relationen liegen. Dies gilt nicht für die zweite Struktur, da die einzige Zweierpotenz,
	welche auch Dreierpotenz ist, die 1 ist. Andererseits sind alle vielfachen von 6 auch Vielfache von 2 und 3, davon gibt es also beliebig viele.\\
	Gewinnstrategie für den Herausforderer im Spiel $G_2(\struc{A},\struc{B})$:
	\begin{enumerate}
	\item H wählt $6\in\mathbb{Z}$ aus \struc{A}
	\item D muss mit 1 aus \struc{B} antworten, da sonst kein lokaler Isomorphismus gegeben wäre, nur 1 in beiden Relationen liegt
	\item Nun wählt H 12, ist erneut in beiden Relationen, darauf kann die Duplikatorin nicht antworten, H gewinnt.
	\end{enumerate}
	In $G_1(\struc{A},\struc{B})$ hat die Duplikatorin eine Gewinnstrategie, führe eine Fallunterscheidung durch:
	\begin{itemize}
	\item H wählt aus \struc{A}, $x \in \mathbb{Z}$
		\begin{itemize}
		\item Es gilt $\neg Px \wedge \neg Qx$: Wähle 0
		\item Es gilt $\neg Px \wedge Qx$: Wähle 3
		\item Es gilt $ Px \wedge \neg Qx$: Wähle 2
		\item Es gilt $ Px \wedge  Qx$: Wähle 1
		\end{itemize}
	\item H wählt aus \struc{B}, $x \in \mathbb{R}$
	\begin{itemize}
		\item Es gilt $\neg Px \wedge \neg Qx$: Wähle 7
		\item Es gilt $\neg Px \wedge Qx$: Wähle 3
		\item Es gilt $ Px \wedge \neg Qx$: Wähle 2
		\item Es gilt $ Px \wedge  Qx$: Wähle 6
	\end{itemize}
	\end{itemize}
\end{adjustwidth}
\textbf{(c)}
\begin{adjustwidth}{20pt}{20pt}
	Es gilt $\struc{A} \equiv_2 \struc{B}$, aber $\struc{A} \not\equiv_3 \struc{B}$. Trennende Formel $\psi=\exists x \exists y \exists z(x \neq y \wedge 1z \wedge Mxyz)$, qr($\psi$)=3.
	Die Formel ist trennend, da in $\mathbb{Z}$ aus $a\cdot b=1$ folgt, dass $a=b=1$, in $\mathbb{Q}$ jedoch nur $b = \frac{1}{a}$, und offensichtlich $a \neq \frac{1}{a}$ falls a nicht die 1
	ist.
Gewinnstrategie für den Herausforderer im Spiel $G_3(\struc{A},\struc{B})$: Herausforderer wählt 1 in \struc{A}, Duplikatorin muss mit 1 antworten, da sonst die einstellgie Relation
zur 1 verletzt. Herausforderer wählt nun 2 in \struc{B}, Dupliaktorin wählt etwas beliebiges in \struc{A}. Herausforderer wählt $\frac{1}{2}$ in \struc{B}, Duplikatorin kann nicht
antworten, da, wie eben begründet, keine zwei Elemente in \struc{A} existieren, diezusammen multipliziert 1 ergeben, es sei denn sie sind beide 1.\\
In $G_2(\struc{A},\struc{B})$ besitzt die dupliaktorin eine Gewinnstrategie: Bemerke zunächst, dass weil nur 2 Züge gespielt werden, die gewählten Elemente nicht verträglich mit der
Relation $M$ sein müssen, da diese dreistellig ist und somit für den lokalen Isomorphiosmus nicht relevant. Falls der Herausforderer die 1 zieht, so muss die Duplikatorin die 1 in der anderen
Struktur wählen, da sonst die dazugehörige Relation verletzt wäre. Wählt der Herausforderer ein Element ungleich 1, so kann die Duplikatorin zu einem beliebigen Element ungleich 1 in der
anderen Struktur ziehen.
\end{adjustwidth}


\section*{Aufgabe 3 (Punkte:\qquad/\pp)}
\textbf{(a)}
\begin{adjustwidth}{20pt}{20pt}
\textbf{(i)}
\begin{adjustwidth}{2em}{0em}\vspace{-\baselineskip}
	Die Theorie einer Struktur ist per Definition vollständig, da diese alle Sätze enthält, welche die Struktur erfüllt. Also enthält die Theorie entweder $\psi$ oder $\neg\psi$ für alle
	für alle FO-Sätze, da die Struktur die Formel entweder erfüllt oder nicht.
\end{adjustwidth}
\textbf{(ii)}
\begin{adjustwidth}{2em}{0em}\vspace{-\baselineskip}
	Sei \struc{A} eine zu $(\mathbb{N},\cdot)$ elementar äquivalente Struktur. Dann ist per Definition von elementarer Äquivalenz $\text{Th}(\struc{A})=\text{Th}((\mathbb{N},\cdot))$. Die
	Theorie der Klasse aller zu $(\mathbb{N},\cdot)$ elementar äquivalenten Strukturen ist der Schnitt der Theorien dieser Strukturen, da diese jedoch alle elementar äquivalent sind ist
	die Theorie der Klasse aller zu $(\mathbb{N},\cdot)$ elementar äquivalenten Strukturen gleich $\text{Th}((\mathbb{N},\cdot))$. Da, wie bereits in (i) festgestellt, die Theorie einer
	Struktur vollständig ist, ist auch diese Theorie vollständig.
\end{adjustwidth}
\textbf{(iii)}
\begin{adjustwidth}{2em}{0em}\vspace{-\baselineskip}
	Die Theorie ist nicht vollständig, betrachte $\struc{B}(aaaaa)$ und $\struc{B}(baaab)$ und den Satz $\psi = \forall x(P_ax)$. Es gilt $\struc{B}(aaaaa) \models \psi$, aber 
	$\struc{B}(baaab) \not\models \psi$, da im Wort $baaab$ nicht jedes Symbol ein $a$ ist. Es ist also $\psi \in \text{Th}(\struc{B}(aaaaa))$, aber $\neg\psi \in \text{Th}(\struc{B}(aaaaa))$.
	Die Theorie der angegebenen Klasse beinhaltet den Schnitt aller Theorien seiner Strukturen, also ist weder $\psi$ noch $\neg\psi$ in der Theorie, also nicht vollständig.
\end{adjustwidth}
\textbf{(iv)}
\begin{adjustwidth}{2em}{0em}\vspace{-\baselineskip}
Da die Graphen keine Kanten haben, ist die Kantenrelation der Graphen leer. Ein Isomorphismus zwischen den Graphen ist also stets verträglich mit der Kantenrelation. Nun existieren beliebig viele
Bijektionen zwischen überabzählbaren Mengen, somit existieren auch Isomorphismen zwischen den Graphen mit überabzählbarer Knotenmenge. Da also alle Graphen dieser Klasse isomorph sind, folgt
mit dem Isomorphielemma direkt, dass diese auch elementar äquivalent sind. Somit entspricht die theorie dieser Klasse gerade der Theorie eines beliebigen Graphen (definiert durch die
entsprechende Struktur). Da die Theorie einer Struktur vollständig ist, ist auch die Theorie in der Aufgabenstellung vollständig. 
\end{adjustwidth}
\end{adjustwidth}
\textbf{(b)}
\begin{adjustwidth}{20pt}{20pt}
\textbf{(i)}
\begin{adjustwidth}{2em}{0em}\vspace{-\baselineskip}

\end{adjustwidth}
\textbf{(ii)}
\begin{adjustwidth}{2em}{0em}\vspace{-\baselineskip}
Die Aussage, dass die Theorie der diskreten linearen Ordnungen ohne Endpunkte (DLOoE) vollständig ist, ist äquivalent zur Aussage, dass alle DLOoE (also alle ihre Modelle) elementar
äquivalent sind. Seien $\struc{A}$ und $\struc{B}$ zwei beliebige DLOoE. Zeige, dass $\struc{A} \equiv \struc{B}$. Dies ist äquivalent zur Aussage, dass für ein beliebiges $m \in
\mathbb{N}$ gilt, dass $\struc{A} \equiv_m \struc{B}$. Mit dem Satz von Ehrenfeucht-Fraissé reicht es zu zeigen, dass die Duplikatorin in $G_m(\struc{A},\struc{B})$ eine Gewinnstrategie
besitzt.\\
Da die Ordnungen keine Endpunkte besitzen, existiert für jedes Element ein größeres und ein kleineres Element. Da die Ordnung diskret ist, besitzt jedes Element ebenfalls einen 
"eindeutigen" Nachfolger und Vorgänger. (Im Folgenden ist mit Nachfolger und Vorgänger stets der kleinste Nachfolger und größte Vorgänger gemeint.)\\
Im ersten Zug wählt der Herausforderer ein Element $z$ aus einer der beiden Strukturen. Die Duplikatorin wähle ein beliebiges Element aus der anderen Struktur. Betrachte nun den i-ten
Zug: Der Herausforderer wählt o.B.d.A ein Element $x$ aus der ersten Struktur (Die Argumentation ist für den Fall, dass aus der zweiten Struktur gewählt wird, analog). Es ergeben sich
zwei Fälle:
\begin{itemize}
\item Sei für $n \leq m$ das gewählte $x$ der $n$-te Nachfolger bzw. Vorgänger für ein bereits in einem vorherigen Zug gewähltem $a_j$ in der ersten Struktur. (Mit "n-ter Nachfolger"
bzw. Vorgänger ist gemeint, dass man n Mal den (eindeutigen) Nachfolger bzw. Vorgänger entlang geht, bis man beim gewählten $x$ ankommt.) Dann wähle die Duplikatorin den $n$-ten
Nachfolger bzw. Vorgänger des zu $a_j$ korrespondierenden Elements $b_j$ in der anderen Struktur, welches im selben Zug gewählt  wurde. (Dies ist möglich, da keine Endpunkte.)
\item Sei nun $x$ kein $n$-ter Nachfolger bzw. Vorgänger für ein $a_j$ (d.h. $x$ liegt von allen Elementen aus mindestens $m+1$ Nachfolger/Vorgänger entfernt). Bezeichne $a_{min}$ das bzgl. der linearen Ordnung kleinste Element der in vorherigen Zügen gewählten Elemente in der ersten Struktur, sei analog $a_{max}$ das Maximum der gewählten Elemente.
Falls $x < a_{min}$ ist, so wählt die Duplikatorin den $(m+1)$-ten Vorgänger vom zu $a_{min}$ korrespondierenden Element $b_{min}$ in der anderen Struktur. Analog wählt die
Duplikatorin, falls $x > a_{max}$ (Dies sind alle möglichen Fälle, sonst wäre $x$ Nachfolger/Vorgänger mit einer Anzahl von weniger als $m$ Schritten),  den $(m+1)$-ten Nachfolger vom zu
$a_{max}$ korrespondierenden Element $b_{max}$ in der anderen Struktur.
\end{itemize}
Hierbei handelt es sich um eine Gewinnstrategie für die Dupliaktorin, denn: Die Idee ist, dass in die Duplikatorin in der anderen Struktur Elemente stets so wählt, dass die Distanz in Bezug auf
Anzahl von Nachfolgern/Vorgängern respektiert werden soll. Dadurch bleibt einerseits die lineare Ordnung erhalten, andererseits kann die diskrete Eigenschaft so nicht vom Herausforderer
zerstört werden. Die Distanzen können aber nur im ersten Fall direkt respektiert werden, da die Distanz insbesondere endlich sit und sich so direkt ind er anderen Struktur umsetzen lässt.
Da aber die Universen der Strukturen überabzählbar sein könnten (?), kann man eventuell nicht direkt bestimmen, der wievielte Nachfolger das gewählte Element in Bezug auf die bereits
gewählten Elemente ist. In dem Fall wählt man einfach einen Nachfolger/Vorgänger, welcher weit genug entfernt ist, sodass der Herausforderer nicht in $m$ Zügen die Elemente immer
enger zwischen zwei vorher ausgewählten Elementen wählen kann, sodass die diskrete Eigenschaft der Ordnung kaputt geht.
\end{adjustwidth}
\end{adjustwidth}

\section*{Aufgabe 4 (Punkte:\qquad/\ppp)}
\textbf{(a)}
\begin{adjustwidth}{20pt}{20pt}
	Zeige die Aussage durch Kontraposition. Es gelte also $\struc{B} \notin \mathcal{K}$. Es ist also $\struc{B} \notin \text{Mod}(\Phi)$ und deswegen $\struc{B} \not\models \Phi$.
	Es existiert also ein Satz $\varphi \in \Phi$, sodass $\struc{B} \not\models \varphi$.\\ %Sei außerdem $m \defgr \text{qr}(\varphi)$.
	Es gilt nun zu zeigen, dass ein $m \in \mathbb{N}$ existiert, sodass für alle $\struc{A} \in \mathcal{K}$ gilt,  dass $\struc{B} \not\equiv_m \struc{A}$. Wähle
	$m \defgr \text{qr}(\varphi)$ und sei $\struc{A} \in \mathcal{K}$ beliebig. Da $\varphi \in \Phi$, folgt, dass $\struc{A} \models \varphi$, da sonst $\struc{A}$ nicht in der
	Modellklasse von $\Phi$ liegen würde. Da nun $\struc{A} \models \varphi$ und $\struc{B} \not\models \varphi$ folgt, dass $\struc{B} \not\equiv_m \struc{A}$. Da $\struc{A}$
	beliebig gewählt war, gilt die zu zeigende Aussage.
\end{adjustwidth}
\textbf{(b)}
\begin{adjustwidth}{20pt}{20pt}
	
\end{adjustwidth}


\end{document}
