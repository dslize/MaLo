\documentclass[11pt, a4paper]{article}

\usepackage{graphicx}
\usepackage{forest}
\usepackage{proof}
\usepackage[utf8]{inputenc}
\usepackage{fancyhdr}
\usepackage{changepage}
\usepackage[onehalfspacing]{setspace}
\usepackage{ragged2e}
\usepackage{ amssymb, amsmath, amsthm, dsfont, marvosym }
\usepackage[width = 18cm, top = 2.5cm, bottom = 3cm]{geometry}
\usepackage{extarrows}
\usepackage{stmaryrd}
\usepackage{enumitem}
% ---------

\newcommand{\myTitleString} {}
\newcommand{\myAuthorString} {}
\newcommand{\mySubTitleString} {}
\newcommand{\myDateString} {}

\newcommand{\myTitle}[1] {\renewcommand {\myTitleString}{#1}}
\newcommand{\mySubTitle}[1] {\renewcommand {\mySubTitleString}{#1}}
\newcommand{\myAuthor}[1] {\renewcommand{\myAuthorString}		{#1}}
\newcommand{\myDate}[1] {\renewcommand{\myDateString}{#1}}

\newcommand{\makeMyTitle}
{
\pagestyle{fancy}
\fancyhead[L]
{
\begin{tabular}{l}
\myTitleString
\\ \mySubTitleString 
\\ \myDateString
\end{tabular}
} 			
\fancyhead[C]{}
\fancyhead[R]{\myAuthorString}
\fancyfoot[C]{\thepage}
}

\setlength{\headheight}{45pt}

\newcommand{\p}{7}
\newcommand{\pp}{11}
\newcommand{\ppp}{8}
\newcommand{\pppp}{} 

\newcommand{\defgl}{\mathrel{=\!\!\mathop:}}
\newcommand{\defgr}{\mathrel{\mathop:\!\!=}}

\newcommand{\struc}[1]{\ensuremath{\mathfrak{#1}}}

\makeatletter
\renewcommand*\env@matrix[1][*\c@MaxMatrixCols c]{%
  \hskip -\arraycolsep
  \let\@ifnextchar\new@ifnextchar
  \array{#1}}
\makeatother
% ---------
%\setlength{\parindent}{0pt}
\begin{document}

\myTitle{\textsc{Mathematische Logik}}
\mySubTitle{Übung 9}
\myDate{26. Juni 2017}
\myAuthor
{
\begin{tabular}{l l}
346532, &Daniel Boschmann\\
348776, &Anton Beliankou	\\
356092, &Daniel Schleiz
\end{tabular}
}
\makeMyTitle

\begin{tabular}{|c|c|c|c|}\hline
   2 & 3 & 4 &$\sum$\\\hline
  	 \qquad/\p & \qquad/\pp & \qquad/\ppp &\qquad/26\\\hline
 \end{tabular}
\hspace*{20pt} {\huge Gruppe \textbf{G}}
\vspace*{30pt}


\section*{Aufgabe 2 (Punkte:\qquad/\p)}
\textbf{(a)}
\begin{adjustwidth}{20pt}{20pt}
	\struc{A} und \struc{B} sind elementar äquivalent, da die Duplikatorin in $G(\struc{A},\struc{B})$ stets eine Gewinnstrategie hat. Da in beiden Strukturen unendlich viele Elemente $x$
	existieren, die in der Relation $P$ liegen. (Und ebenso welche, die nicht in der Relation liegen.) Als Gewinnstrategie könnte die Duplikatorin zum Beispiel für eine nicht-negative Zahl aus
	der ersten Struktur (vom Herausforderer gewählt) aus der zweiten Struktur Zahlen in der Reihenfolge -1,-2,-3,... geben (falls eine solche Zahl in einem anderen Schritt vom erausforderer bereits gewählt wurde, wird
	einfach die nächste Zahl der Folge gewählt), bei einer negativen Zahl eben positive Zahlen in der zweiten Struktur in der Folge 1,2,3,... (Analoge Strategie, falls der Herausforderer aus
	der zweiten Struktur wählt, also genau andersrum.) Somit sind die Strukturen elementar äquivalent.
	
\end{adjustwidth}
\textbf{(b)}
\begin{adjustwidth}{20pt}{20pt}
	Es gilt $\struc{A} \equiv_1 \struc{B}$, aber $\struc{A} \not\equiv_2 \struc{B}$. Trennende Formel: $\psi = \exists x \exists y(x \neq y \wedge Px \wedge Qx \wedge Py \wedge Qy)$, es ist
	qr($\psi$)=2. Die Formel sagt also aus, dass es mindestens zwei Elemente gibt, welche in beiden Relationen liegen. Dies gilt nicht für die zweite Struktur, da die einzige Zweierpotenz,
	welche auch Dreierpotenz ist, die 1 ist. Andererseits sind alle vielfachen von 6 auch Vielfache von 2 und 3, davon gibt es also beliebig viele.\\
	Gewinnstrategie für den Herausforderer im Spiel $G_2(\struc{A},\struc{B})$:
	\begin{enumerate}
	\item H wählt $6\in\mathbb{Z}$ aus \struc{A}
	\item D muss mit 1 aus \struc{B} antworten, da sonst kein lokaler Isomorphismus gegeben wäre, nur 1 in beiden Relationen liegt
	\item Nun wählt H 12, ist erneut in beiden Relationen, darauf kann die Duplikatorin nicht antworten, H gewinnt.
	\end{enumerate}
	In $G_1(\struc{A},\struc{B})$ hat die Duplikatorin eine Gewinnstrategie, führe eine Fallunterscheidung durch:
	\begin{itemize}
	\item H wählt aus \struc{A}, $x \in \mathbb{Z}$
		\begin{itemize}
		\item 
		\end{itemize}
	\item H wählt aus \struc{B}, $x \in \mathbb{R}$
	\end{itemize}
\end{adjustwidth}
\textbf{(c)}
\begin{adjustwidth}{20pt}{20pt}
	Es gilt $\struc{A} \equiv_2 \struc{B}$, aber $\struc{A} \not\equiv_3 \struc{B}$. Trennende Formel $\psi=\exists x \exists y \exists z(x \neq y \wedge 1z \wedge Mxyz)$, qr($\psi$)=3.
	Die Formel ist trennend, da in $\mathbb{Z}$ aus $a\cdot b=1$ folgt, dass $a=b=1$, in $\mathbb{Q}$ jedoch nur $b = \frac{1}{a}$, und offensichtlich $a \neq \frac{1}{a}$ falls a nicht die 1
	ist.
\end{adjustwidth}


\section*{Aufgabe 3 (Punkte:\qquad/\pp)}
\textbf{(a)}
\begin{adjustwidth}{20pt}{20pt}
\textbf{(i)}
\begin{adjustwidth}{2em}{0em}\vspace{-\baselineskip}
	Die Theorie einer Struktur ist per Definition vollständig, da diese alle Sätze enthält, welche die Struktur erfüllt. Also enthält die Theorie entweder $\psi$ oder $\neg\psi$ für alle
	für alle FO-Sätze, da die Struktur die Formel entweder erfüllt oder nicht.
\end{adjustwidth}
\textbf{(ii)}
\begin{adjustwidth}{2em}{0em}\vspace{-\baselineskip}
	Sei \struc{A} eine zu $(\mathbb{N},\cdot)$ elementar äquivalente Struktur. Dann ist per Definition von elementarer Äquivalenz $\text{Th}(\struc{A})=\text{Th}((\mathbb{N},\cdot))$. Die
	Theorie der Klasse aller zu $(\mathbb{N},\cdot)$ elementar äquivalenten Strukturen ist der Schnitt der Theorien dieser Strukturen, da diese jedoch alle elementar äquivalent sind ist
	die Theorie der Klasse aller zu $(\mathbb{N},\cdot)$ elementar äquivalenten Strukturen gleich $\text{Th}((\mathbb{N},\cdot))$. Da, wie bereits in (i) festgestellt, die Theorie einer
	Struktur vollständig ist, ist auch diese Theorie vollständig.
\end{adjustwidth}
\textbf{(iii)}
\begin{adjustwidth}{2em}{0em}\vspace{-\baselineskip}
	Die Theorie ist nicht vollständig, betrachte $\struc{B}(aaaaa)$ und $\struc{B}(baaab)$ und den Satz $\psi = \forall x(P_ax)$. Es gilt $\struc{B}(aaaaa) \models \psi$, aber 
	$\struc{B}(baaab) \not\models \psi$, da im Wort $baaab$ nicht jedes Symbol ein $a$ ist. Es ist also $\psi \in \text{Th}(\struc{B}(aaaaa))$, aber $\neg\psi \in \text{Th}(\struc{B}(aaaaa))$.
	Die Theorie der angegebenen Klasse beinhaltet den Schnitt aller Theorien seiner Strukturen, also ist weder $\psi$ noch $\neg\psi$ in der Theorie, also nicht vollständig.
\end{adjustwidth}
\textbf{(iv)}
\begin{adjustwidth}{2em}{0em}\vspace{-\baselineskip}
Da die Graphen keine Kanten haben, ist die Kantenrelation der Graphen leer. Ein Isomorphismus zwischen den Graphen ist also stets verträglich mit der Kantenrelation. Nun existieren beliebig viele
Bijektionen zwischen überabzählbaren Mengen, somit existieren auch Isomorphismen zwischen den Graphen mit überabzählbarer Knotenmenge. Da also alle Graphen dieser Klasse isomorph sind, folgt
mit dem Isomorphielemma direkt, dass diese auch elementar äquivalent sind. Somit entspricht die theorie dieser Klasse gerade der Theorie eines beliebigen Graphen (definiert durch die
entsprechende Struktur). Da die Theorie einer Struktur vollständig ist, ist auch die Theorie in der Aufgabenstellung vollständig. 
\end{adjustwidth}
\end{adjustwidth}
\textbf{(b)}
\begin{adjustwidth}{20pt}{20pt}
	
\end{adjustwidth}

\section*{Aufgabe 4 (Punkte:\qquad/\ppp)}
\textbf{(a)}
\begin{adjustwidth}{20pt}{20pt}
	
\end{adjustwidth}
\textbf{(b)}
\begin{adjustwidth}{20pt}{20pt}
	
\end{adjustwidth}


\end{document}
